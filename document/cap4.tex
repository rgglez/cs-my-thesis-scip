\chapter[Pruebas del Sistema y sus Resultados.]{Pruebas del Sistema y sus Resultados.}
\label{Capitulo:pruebas}

%----------------------------------------------------------------------------

\fancyhead[RE]{\slshape SCIP - Cap. \thechapter~ - Pruebas del Sistema y sus %%@
Resultados.}
\fancyhead[LO]{\slshape Rodolfo Gonz\'alez Gonz\'alez.}
\fancyhead[LE,RO]{\thepage}

%----------------------------------------------------------------------------

\section{Introducci\'on.}
\label{Section:intropruebas}

En \'este cap\'{\i}tulo se presentan las pruebas efectuadas al sistema, y sus resultados. El %%@
software de prueba desarrollado es b\'asico y solo fue creado con fines de demostraci\'on del %%@
hardware.

%----------------------------------------------------------------------------

\section{Pruebas del sistema.}
\label{Section:pruebas}

Se realizaron las siguientes pruebas en el sistema, para demostrar la capacidad de ejecutar otro %%@
programa al mismo tiempo que la PC anfitriona, y comunicar a SCIP con dicha computadora:

\begin{enumerate}
\item Comunicaci\'on unidireccional de SCIP a la PC anfitriona.
\item Comunicaci\'on bidireccional sincronizada por software.
\end{enumerate}

Para estos programas se utilizo Turbo Assembler y Turbo C, ambos de Borland Int.

%----------------------------------------------------------------------------

\subsubsection{Comunicaci\'on unidireccional de SCIP a la PC anfitriona.}
\label{Subsubsection:prueba1}

Esta prueba consisti\'o en que la tarjeta SCIP enviara datos a la computadora anfitriona %%@
mediante el 8255 programado en Modo 0 (Ver Subsecci\'on~\ref{Subsection:comunicacion}), y que %%@
un programa en aquella computadora los recibiera y desplegara en pantalla.

El siguiente c\'odigo en ensamblador del 8088 fue programado y grabado en la memoria EPROM %%@
correspondiente al bloque 6 de memoria de SCIP (Ver Tabla \ref{Tabla:decodifmem}):

\begin{listing}{1}
.MODEL SMALL

CODIGO SEGMENT 'Code'
   ASSUME CS:CODIGO, DS:CODIGO, ES:NOTHING, SS:NOTHING

   ORG 0F0100h

PROG PROC NEAR    
   XOR AX,AX
   MOV DX,3C7h    ; Direccion del pto. de control del 8255
   MOV AL,8Ah     ; Modo 0, A=Salida, B=Entrada, 
                  ; PC03 = Salida, PC47 = Entrada
   OUT DX,AL      ; Programar 8255 

   MOV DX,3C4h    ; Data Bus -> Port A
   MOV AL,55h     ; Valor a enviar a la PC anfitriona
   
 WRITE:           ; Ciclo infinito de escritura
   OUT DX,AL     
   JMP WRITE
   
   ORG 0FFFF0h    ; Inicio del programa F000:FFF0

 START:           ; Inicializacion del CS
   MOV AX,0F000h  
   MOV CS,AX
   JMP PROG       ; Salto a la rutina principal
PROG ENDP
CODIGO ENDS
   END PROG
\end{listing}

En es\-te lis\-ta\-do se a\-pre\-cia el blo\-que de ini\-cia\-li\-za\-ci\-\'on des\-cri\-%%@
to en la Secci\'on~\ref{Section:consideraciones}, en las l\'{\i}neas 23 a 28.

En las l\'{\i}neas 8 a 21 del listado se muestra en c\'odigo principal del programa, el cual %%@
realiza lo siguiente:

\begin{enumerate}
\item Programar el 8255 en Modo 0 (como se describe Subsecci\'on %%@
\ref{Subsection:comunicacion}), en las l\'{\i}neas 9 a 14.
\item Enviar en un ciclo infinito el valor 55h al puerto 3C4h (puerto de salida de SCIP, ver %%@
Tabla~\ref{Tabla:program8255}), en las l\'{\i}neas 16 a 21.
\end{enumerate}

En programa que se ejecut\'o en la computadora anfitriona fue realizado en lenguaje C, y su %%@
listado se presenta a continuaci\'on:

\begin{listing}{1}
#include <stdio.h>
#include <dos.h>

int data;

void main(void) 
{ 
    do {
       data = inportb(0x300);  /* Leer byte del pto. */
       printf("%c",data);      /* Escribir byte a pantalla */
   } while (1);
}
\end{listing}

Lo que realiza este programa es un ciclo infinito (l\'{\i}neas 8 a 11) en donde se lee un byte %%@
del puerto 300h (puerto de lectura de SCIP por parte de la PC anfitriona) e imprimirlo a %%@
pantalla (l\'{\i}neas 9 y 10). Entonces en pantalla se desplegar\'a el valor enviado ('U' = %%@
55h).

%----------------------------------------------------------------------------

\subsubsection{Comunicaci\'on bidireccional sincronizada por software.}
\label{Subsubsection:prueba2}

Esta prueba consisti\'o en que los caracteres tecleados por el usuario en la computadora %%@
anfitriona fueran enviados a la tarjeta SCIP, la cual los reenviara a aquella computadora, y %%@
fueran desplegados en pantalla, todo esto sincronizando las transferencias con caracteres de %%@
control, como se explica a continuaci\'on.

\begin{listing}{1}
#include <stdio.h>
#include <dos.h>
#include <conio.h>

int data;
char c = '\0';

void main(void)
{
   clrscr();

   c = getch();             /* Leer caracter */
   while (c != 'T') {       /* Termina cuando se lee T */
      outportb(0x301,'T');  /* Enviar caracter de inicio */

      do {                        
         data = inportb(0x300);/* Esperar por caracter de reconoci- */
     } while (data != 'R'); /* miento enviado por SCIP (R) */


      outportb(0x301,c);    /* Enviar a SCIP caracter leido */
      delay(1);       
      data = inportb(0x300);/* Leer caracter enviado por SCIP */

      printf("%c",data);    /* Imprimir a pantalla el caracter */
      c = getch();          /* Leer nuevo caracter de teclado */
   }
}
\end{listing}

El programa cuyo listado se presenta arriba esta realizado en lenguaje C y se ejecuta en la PC %%@
anfitriona. Lo que realiza es lo siguiente:

\begin{enumerate}
\item Leer un car\'acter\footnote{Un car\'acter corresponde a un byte (8 bits).} del %%@
teclado (l\'{\i}nea 12).
\item Si el car\'acter le\'{\i}do es una T, entonces termina el programa (l\'{\i}nea 13).
\item Enviar a SCIP por el puerto 301h el car\'acter T que indica ``Inicio de Transmisi\'on'' %%@
(l\'{\i}nea 14).
\item Esperar en un ciclo a que SCIP envie en el puerto 300h el car\'acter R que indica %%@
``Transmisi\'on Reconocida'' (l\'{\i}neas 16 a 18).
\item Enviar a SCIP el car\'acter le\'{\i}do por el teclado (l\'{\i}nea 21). 
\item Leer de SCIP el car\'acter que la tarjeta env\'{\i}a, que es el mismo que el programa %%@
en la computadora anfitriona le envi\'o (l\'{\i}nea 23).
\item Imprimir en pantalla el car\'acter le\'{\i}do (l\'{\i}nea 25).
\item Leer nuevo car\'acter del teclado y repetir el ciclo desde el paso 2 (l\'{\i}nea 26).
\end{enumerate}

Como se dijo en la Subsecci\'on \ref{Subsection:decodif} los puertos utilizados son el 300h y %%@
301h (entrada y salida, respectivamente). Los caracteres usados para enviar mensajes son en este %%@
ejemplo T y R, pero pudieran haberse utilizado otros cualesquiera.

El programa en ensamblador ejecutado por SCIP se presenta a con\-ti\-nua\-ci\-\'on:

\begin{listing}{1}
.MODEL SMALL

CODIGO SEGMENT 'Code'
   ASSUME CS:CODIGO, DS:CODIGO, ES:NOTHING, SS:NOTHING

   ORG 0F0100h

PROG PROC NEAR
   XOR AX,AX
   XOR BX,BX

   MOV DX,3C7h  ; Direccion del pto. de control del 8255 (A0=A1=1)
   MOV AL,8Ah   ; Modo 0, Pto. A = Salida, Pto. B = Entrada, 
                ; PC03 = Salida, PC47 = Entrada
   OUT DX,AL    ; Programar 8255

 CICLO:
   MOV DX,3C5h  ; Port B -> Data Bus

 LEE1:          ; Esperar comando de inicio de transmision (T)
   IN  AL,DX 
   CMP AL,'T'
   JNE LEE1

   MOV AL,'R'   ; Enviar al anfitrion un comando de reconocimiento (R)
   MOV DX,3C4h  ; Data Bus -> Port A
   OUT DX,AL

   MOV DX,3C5h  ; Port B -> Data Bus

 LEE2:          ; Esperar primer caracter del anfitrion
   IN  AL,DX
   CMP AL,'T'
   JE LEE2

   MOV DX,3C4h  ; Data Bus -> Port A
   OUT DX,AL    ; Enviar caracter recibido de regreso a anfitrion

   JMP CICLO    ; Reiniciar el ciclo

   ORG 0FFFF0h

 START:         ; Inicializar CS y saltar al inicio del programa
   MOV AX,0F000h
   MOV CS,AX
   JMP PROG
PROG ENDP
CODIGO ENDS
   END PROG
\end{listing}

Este programa realiza lo siguiente, en un ciclo infinito:

\begin{enumerate}
\item En las l\'{\i}neas 12 a 15 se programa al 8255 para trabajar en Modo 0.
\item Esperar en un ciclo leyendo al puerto 3C5h (Puerto B a Bus de Datos) a que se reciba el %%@
car\'acter de Inicio de Transmisi\'on (T) (l\'{\i}neas 18 a 23).
\item Enviar un car\'acter R para indicar a la computadora anfitriona que se recibi\'o el %%@
car\'acter T (l\'{\i}neas 25 a 27).
\item Esperar en un ciclo a que llegue el car\'acter de la computadora anfitriona %%@
(l\'{\i}neas 29 a 34).
\item Enviar a la computadora anfitriona el car\'acter recibido (l\'{\i}neas 36 y 37).
\end{enumerate}

En ambos casos el programa en ejecuci\'on en SCIP puede reiniciarse aplicando la se\~nal de %%@
RESET mediante el bot\'on ubicado en la tarjeta.

%----------------------------------------------------------------------------

\section{Resultados de las pruebas.}
\label{Section:resultados}

Los programas de prueba desarrollados y presentados en el documento permiten que la computadora %%@
y SCIP se comuniquen sincronizadamente, y que un programa se este ejecutando en SCIP al mismo %%@
tiempo que otro se ejecuta en la computadora anfitriona. Los resultados fueron considerados %%@
satisfactorios.

%----------------------------------------------------------------------------
