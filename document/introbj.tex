\vspace*{20pt}
\begin{flushleft}
{\LARGE\bf Introducci\'on y Objetivos.}
\end{flushleft}
\vskip 50pt
\label{Chapter:introbj}

%----------------------------------------------------------------------------

\subsubsection{Introducci\'on.}
\label{Subsubsection:intro}

El {\bf Sistema de Coprocesamiento de Informaci\'on para %%@
PC\footnote{Computadora Personal, por sus siglas en %%@
ingl\'es.}} (SCIP) consiste en una tarjeta de expansi\'on de %%@
8 bits para un bus AT de una PC-IBM o compatible cuya tarea %%@
es realizar el procesamiento de informaci\'on al mismo tiempo %%@
que la CPU de la computadora anfitriona ejecuta otra tarea, %%@
llam\'andose a esto coprocesamiento. El sistema consiste %%@
b\'a\-si\-ca\-men\-te de un  procesador 8088, un %%@
coprocesador a\-rit\-m\'e\-ti\-co 8087, me\-mo\-ria EPROM %%@
con ru\-ti\-nas b\'a\-si\-cas, chips de me\-mo\-ria RAM %%@
es\-t\'a\-ti\-ca de tra\-ba\-jo y una EPROM pa\-ra ru\-%%@
ti\-nas del u\-sua\-rio, aunado a la l\'ogica adicional de %%@
soporte de estos componentes.

La funci\'on de la tarjeta coprocesadora es la de realizar el %%@
coprocesamiento
de informaci\'on de manera que el procesador principal de la %%@
computadora delegue determinadas funciones en esta tarjeta. %%@
El sistema planteado es multiprop\'ositos, y puede ser %%@
programado por el usuario.

El procesador principal ver\'a al procesador de la tarjeta de %%@
expansi\'on como un puerto de entrada/salida.

El proyecto surge ante la necesidad de realizar un %%@
procesamiento r\'apido de 
datos anal\'ogicos adquiridos por medio de alg\'un %%@
dispositivo de conversi\'on anal\'ogico-digital, pero fue %%@
extendido a un sistema de coprocesamiento de prop\'osito %%@
general. El sistema de coprocesamiento es \'util ya que %%@
permite que el procesador principal de la PC delegue %%@
determinadas operaciones a la tarjeta coprocesadora. En el %%@
sistema presentado no se desarrolla la fase de adquisici\'on %%@
de datos, ya que esto no entr\'o dentro de los objetivos del %%@
presente trabajo.

%----------------------------------------------------------------------------

\subsubsection{Objetivos Generales del Proyecto.}
\label{Subsubsection:objetivosgenerales}

Se plantea como objetivo general el desarrollo de una tarjeta %%@
de expansi\'on para una PC-IBM o compatible que realice %%@
coprocesamiento de informaci\'on.

\subsubsection{Objetivos Espec\'{\i}ficos del Proyecto.}
\label{Subsubsection:objetivosespecificos}

Se plantearon los siguientes objetivos espec\'{\i}ficos:

\begin{enumerate}
\item El dise\~no y desarrollo de una tarjeta de %%@
expansi\'on para computadoras PC-IBM y compatibles con bus AT %%@
que realice la ejecuci\'on de un programa al mismo tiempo que %%@
la CPU de la computadora anfitriona ejecuta otro programa.
\item La utilizaci\'on del procesador Intel 8088 y su %%@
l\'ogica de soporte para la realizaci\'on del procesamiento, %%@
incluyendo memorias RAM y ROM.
\item La utilizaci\'on del Controlador de Perif\'ericos %%@
Programable 8255 para efectuar la comunicaci\'on %%@
bidireccional entre la tarjeta SCIP y la computadora %%@
anfitriona.
\item El desarrollo de programas b\'asicos que permitan %%@
comunicar a la tarjeta con la computadora.
\end{enumerate}

%----------------------------------------------------------------------------
