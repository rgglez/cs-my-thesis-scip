\chapter[Microprocesadores.]{Microprocesadores.}
\label{Capitulo:Microprocesadores}

%----------------------------------------------------------------------------

\fancyhead[RE]{\slshape SCIP - Cap. \thechapter~ - Microprocesadores.}
\fancyhead[LO]{\slshape Rodolfo Gonz\'alez Gonz\'alez.}
\fancyhead[LE,RO]{\thepage}

%----------------------------------------------------------------------------

\section{Introducci\'on.}
\label{Seccion:IntroCap1}

En este cap\'{\i}tulo se hace una breve rese\~na hist\'orica de los microprocesadores y %%@
coprocesadores y se presenta una introducci\'on a los microprocesadores Intel 8088 y 8086.

%----------------------------------------------------------------------------

\section{Generalidades.}
\label{Seccion:Generalidades}

%----------------------------------------------------------------------------

\subsection{Evoluci\'on del Microprocesador.}
\label{Subseccion:Evolucion}

El microprocesador es la parte fundamental de una microcomputadora. Un microprocesador es la %%@
{\em Unidad de Procesamiento Central} ({\it CPU\/} por sus siglas en ingl\'es), que controla %%@
la operaci\'on de la computadora. Toma una instrucci\'on codificada en binario de la memoria, %%@
decodifica las instrucciones en series de acciones simples, y maneja dichas acciones. La CPU %%@
contiene una {\em Unidad Aritm\'etica y L\'ogica} (ALU) que realiza diversas operaciones a\-%%@
rit\-m\'e\-ti\-cas y l\'ogicas sobre palabras binarias. La CPU tambi\'en contiene un {\em %%@
apuntador a instrucciones\/} (IP o {\it instruction pointer\/}) usado para cargar la %%@
direcci\'on de la siguiente instrucci\'on o dato de la memoria, registros de prop\'osito %%@
general, y circuiter\'{\i}a para controlar las se\~nales del bus \cite{Hall}.

Hist\'oricamente, el primer microprocesador fue el Intel 4004, aparecido en Noviembre de 1971. %%@
Este era un microprocesador de 4 bits ideado para ser utilizado en una calculadora \cite{Hall}. %%@
Este procesador funcionaba a una frecuencia de 108Khz \cite{Micro:Hist}.

Posteriormente, Intel producir\'{\i}a el microprocesador 8008, usado como controlador de %%@
terminales. El sucesor de este fue el Intel 8080, aparecido en 1974, que ten\'{\i}a un bus de %%@
datos de 8 bits, y un bus de direcciones de 16 bits, teniendo internamente 7 registros de 8 %%@
bits. Con este microprocesador aparecieron los puertos de entrada y salida \cite{Micro:Hist}.

\begin{table}[!htb]
\centering
\begin{tabular}{|l|} \hline
{\tiny Intel 4004, 4040} \\ \hline
{\tiny Intel 8008, 8080, 8085} \\ \hline
{\tiny Intel 8048, 8051, 8052} \\ \hline
{\tiny Intel 80x86, Pentium, AMD K5/K6, Cyrix M1, Nx586} \\ \hline
{\tiny Intel 80960} \\ \hline
{\tiny Intel 80860} \\ \hline
{\tiny Intel i432} \\ \hline
{\tiny Motorola MC14500B} \\ \hline
{\tiny Motorola 680x, 6809, Hitachi 6309} \\ \hline
{\tiny Motorola 680x0} \\ \hline
{\tiny Motorola 88000} \\ \hline
{\tiny Motorola DSP96002/DSP56000} \\ \hline
{\tiny AMD 29000} \\ \hline
{\tiny Zilog Z-80, Z-280} \\ \hline
{\tiny Zilog Z-8000, Z80000} \\ \hline
{\tiny Fairchild F8} \\ \hline
{\tiny Fairchild 9440} \\ \hline
\end{tabular} 
\caption{Algunos microprocesadores.} 
\label{Micros1}
\end{table}

Un avance sobre el Intel 8080 fue el 8085 (1976), el cual tiene dos instrucciones para %%@
deshabilitar o habilitar tres terminales de interrupci\'on, y la terminal de entrada y salida %%@
serial. Estas instrucciones son RIM y SIM \cite{CEKIT}. Tambi\'en ten\'{\i}a el generador de %%@
reloj y el controlador de bus integrados. Otras compa\~nias fabricaron pronto microprocesadores %%@
de 8 bits, dando inicio la era de las microcomputadoras; entre estos se encuentran el Motorola %%@
MC6800, el MOS Technology 6502 (usado como CPU en la microcomputadora Apple II \cite{Hall}) y %%@
el Zilog Z80 (utilizado como CPU en la microcomputadora Radio Shack TRS-80 \cite{Hall}).

\begin{table}[!htb]
\centering
\begin{tabular}{|l|} \hline
{\tiny Fairchild/Intergraph Clipper} \\ \hline
{\tiny National Semiconductor SC/MP (and COP)} \\ \hline 
{\tiny National Semiconductor 320xx, Swordfish} \\ \hline
{\tiny TI 9900} \\ \hline
{\tiny TI TMS320Cx0} \\ \hline
{\tiny MIPS R2000, R3000, R4000, R5000, R8000, R10000} \\ \hline
{\tiny MOS Technologies 650x, Western Design Center 65816} \\ \hline
{\tiny Microchip Technology PIC 16x} \\ \hline
{\tiny RCA 1802} \\ \hline
{\tiny Ferranti F100-L} \\ \hline
{\tiny Western Digital MCP-1600} \\ \hline
{\tiny Signetics 2650} \\ \hline
{\tiny Siemens 80C166} \\ \hline
{\tiny MISC M17} \\ \hline
{\tiny Rekursiv} \\ \hline
{\tiny AT\&T CRISP/Hobbit} \\ \hline
\end{tabular} 
\caption{Algunos microprocesadores ({\it continuaci\'on\/}).} 
\label{Micros2}
\end{table}

\begin{table}[!hbt]
\centering
\begin{tabular}{|l|} \hline
{\tiny PDP-8/Intersil 6100} \\ \hline
{\tiny PDP-11} \\ \hline
{\tiny Data General NOVA/MN601, Eclipse} \\ \hline
{\tiny IBM/Motorola POWER/PowerPC} \\ \hline
{\tiny IBM 801, ROMP} \\ \hline
{\tiny IBM System/360/370/390} \\ \hline
{\tiny TRON} \\ \hline
{\tiny Hitachi SuperH} \\ \hline 
{\tiny SPARC} \\ \hline
{\tiny HP PA-RISC} \\ \hline
{\tiny ARM} \\ \hline
{\tiny Patriot Scientific ShBoom} \\ \hline
{\tiny DEC VAX} \\ \hline
{\tiny DEC Alpha} \\ \hline
{\tiny CDC 6600/7600} \\ \hline
{\tiny Berkeley RISC} \\ \hline
\end{tabular}
\caption{Arquitecturas.} 
\label{Arquitecturas}
\end{table}

Posteriormente han aparecido muchos m\'as microprocesadores, algunos de los cu\'ales se %%@
presentan en las Tablas~\ref{Micros1} y \ref{Micros2} \cite{Micro:Hist}, y sus usos se han %%@
ido extendiendo en otros campos, no solo como CPU's de microcomputadoras. Hall \cite{Hall} %%@
define dos direcciones de evoluci\'on en los microprocesadores (ver Subsecciones %%@
\ref{Subseccion:controladores} y \ref{Subseccion:propogen}). As\'{\i} mismo han aparecido %%@
diversas arquitecturas de procesadores, algunas de las cuales se presentan en la Tabla %%@
\ref{Arquitecturas} \cite{Micro:Hist}.

%----------------------------------------------------------------------------

\subsection{Controladores Dedicados.}
\label{Subseccion:controladores}

Este tipo de dispositivos tienen aplicaci\'on en los m\'as diversos campos, desde el control de %%@
hornos de microondas hasta sistemas de rehabilitaci\'on para personas discapacitadas. Entre los %%@
primeros microcontroladores aparecidos en el mercado est\'an el Texas Instruments TMS-1000 y el %%@
Intel 8048. Actualmente existen microcontroladores como los Intel 8051, 8096 y los Motorola %%@
MC6801 y MC6811 \cite{Hall}.

%----------------------------------------------------------------------------

\subsection{CPU's de Prop\'osito General.}
\label{Subseccion:propogen}

Es\-te es el cam\-po de a\-pli\-ca\-ci\'on m\'as ex\-ten\-so de los mi\-cro\-pro\-ce\-%%@
sa\-do\-res, los cu\'a\-les son u\-sa\-dos en mi\-cro\-com\-pu\-ta\-do\-ras m\'as po\-%%@
de\-ro\-sas que la ma\-yo\-r\'{\i}a de las mi\-ni\-com\-pu\-ta\-do\-ras an\-te\-rio\-res %%@
a su a\-pa\-ri\-ci\'on. 

%----------------------------------------------------------------------------

\section{Introducci\'on a los mi\-cro\-pro\-ce\-sa\-do\-res Intel 8088 y 80x86.}
\label{Seccion:8088y8086}

Los microprocesadores Intel fueron elegidos por IBM para sus microcomputadoras por razones de %%@
mercadotecnia y bajo costo comparativamente a otros dispositivos, as\'{\i} como por razones de %%@
derechos industriales \cite{Micro:Hist}. La primera computadora personal de la IBM %%@
aparecer\'{\i}a el 12 de agosto de 1981, usando el chip Intel 8088 \cite{SADYR}. Esta %%@
m\'aquina marc\'o el despegue mercantil de las computadoras multiprop\'osito para el hogar y la %%@
oficina ({\it Computadoras Personales, o PC's\/}). Las {\it PC's} comenzaron a explorar la %%@
arquitectura abierta, con la posibilidad de a\~nadir {\em tarjetas de expansi\'on } en la %%@
tarjeta principal de la computadora, en ranuras ({\it slots\/}) provistas para tal fin en el %%@
{\em bus de expansi\'on\/}. Actualmente estas tarjetas abarcan una amplia gama de %%@
aplicaciones, que va desde tarjetas de video, MODEM's\footnote{Dispositivos Moduladores-%%@
Demoduladores que permiten la comunicaci\'on telef\'onica mediante la computadora.} y tarjetas %%@
controladoras de diversos dispositivos perif\'ericos, hasta tarjetas de digitalizaci\'on o %%@
especializadas en prop\'ositos cient\'{\i}ficos.

%----------------------------------------------------------------------------

\subsection{Coprocesadores matem\'aticos.}
\label{Subseccion:coprocesadores}

Frecuentemente se asocian a los chips Intel 80x86 circuitos integrados, llamados coprocesadores, %%@
cuyo prop\'osito es llevar a cabo operaciones es\-pe\-cia\-li\-za\-das a alta velocidad, %%@
liberando a la CPU principal de determinada tarea \cite{Godfrey}. El caso m\'as com\'un es el %%@
de los coprocesadores de punto flotante, encargados de realizar operaciones de punto flotante a %%@
alta velocidad. Hist\'oricamente uno de los primeros coprocesadores matem\'aticos fue el AMD %%@
9511. 

\begin{table}[!htb] 
\centering
\begin{tabular}{|l|l|} \hline
{\bf Microprocesador} & {\bf Coprocesador} \\ \hline
8088                  & 8087               \\ \hline
8086                  & 8087               \\ \hline
80286                 & 80287              \\ \hline
80386                 & 80387              \\ \hline
\end{tabular} 
\caption{Microprocesadores y Coprocesadores en el caso de computadoras basadas en Intel %%@
8088/80x86.} 
\label{tabla:micro-y-copro}
\end{table}

Los coprocesadores matem\'aticos utilizados dependen del microprocesador principal de la %%@
computadora. Estas dependencias se presentan en la Tabla~\ref{tabla:micro-y-copro} %%@
\cite{Godfrey}.

El uso de los coprocesadores no se limita a las operaciones de punto flotante. Existen %%@
coprocesadores con aplicaci\'on en adaptadores gr\'aficos ({\it coprocesadores %%@
gr\'aficos\/}), como los coprocesadores gr\'aficos 34010 o 34020 de Texas Instruments %%@
\cite{PC-Mag:V4N2}.

Las es\-pe\-ci\-fi\-ca\-cio\-nes del co\-pro\-ce\-sa\-dor ma\-te\-m\'a\-ti\-co 8087 se %%@
pre\-sen\-tan bre\-ve\-men\-te en el A\-p\'en\-di\-ce~\ref{Apendice:chips} y m\'as am\-%%@
plia\-men\-te en \cite{Intel:Micro}, mientras que en la Secci\'on~\ref{Section:copronum} se %%@
presenta una breve des\-crip\-ci\'on del mismo.

%----------------------------------------------------------------------------

\subsection{Ar\-qui\-tec\-tu\-ra in\-ter\-na de los\\mi\-cro\-pro\-ce\-sa\-do\-res %%@
Intel 8088 y 8086.}
\label{Subseccion:arquitectura}

Las microcomputadoras IBM han estado basadas en los microprocesadores Intel 8088 y 80x86. Se les %%@
llama {\em compatibles con IBM} a las microcomputadoras con organizaci\'on y arquitecturas %%@
compatibles con las de dicha empresa. En el Ap\'endice \ref{Apendice:chips} se presenta una %%@
breve descripci\'on del 8088 y su hoja t\'ecnica tomada de \cite{Intel:Micro}.

\begin{figure}[!htb]
\vskip 5mm
\special{center
         1-1.gif,                
         \the\hsize 78mm}
\vskip 90mm
\caption{Sistema 8086/8088 b\'asico en modo m\'{\i}nimo.} 
\label{modomin}
\end{figure}

\begin{figure}[!hbt]
\vskip 5mm
\special{center
         1-8.gif,                
         \the\hsize 78mm}
\vskip 90mm
\caption{Configuraci\'on ``Modo M\'aximo'' del 8088.} 
\label{modomax}
\end{figure}

Las primeras microcomputadoras de IBM utilizaban el microprocesador Intel 8088 y las {\em %%@
tarjetas madres} o {\em motherboards} de ambas (8088 y 8086) son similares \cite{Godfrey}. El %%@
esquema de conexi\'on en modo m\'{\i}nimo se presenta en la Figura~\ref{modomin} y para el %%@
caso del modo m\'aximo se presenta en la Figura~\ref{modomax} \cite{Hall}. 

Esencialmente el microprocesador 8088 y el 8086 son iguales en lo re\-fe\-ren\-te a sus %%@
terminales. Ambos son microprocesadores con registros de 16 bits.

El Intel 8088 es un microprocesador de tecnolog\'{\i}a HMOS, en un circuito integrado de 40 %%@
terminales. Su conjunto de instrucciones es compatible con el 8086. La tarjeta principal de una %%@
computadora basada en 8088 u 8086 tiene un conector para el coprocesador matem\'atico 8087. En %%@
la PC original hay 5 ranuras de expansi\'on, n\'umero que crece en la XT a 8 (cabe se\~nalar %%@
que en las microcomputadoras actuales el n\'umero de ranuras de expansi\'on puede variar de un %%@
modelo a otro). El control de interrupciones lo lleva a cabo el Intel 8259A, un controlador de %%@
interrupciones programable. El Intel 8255A es la interfaz al teclado \cite{Godfrey}. El Intel %%@
8272A es el circuito integrado utilizado como controlador de las unidades de discos flexibles. %%@
El Intel 8037A es el controlador de acceso directo a memoria (DMA) usado para manejar las %%@
transferencias en bloque de entrada y salida de datos, y el Intel 8253, que es un contador y %%@
temporizador usado como contador de eventos \cite{Godfrey}. El circuito encargado de generar %%@
los pulsos de reloj para el microprocesador y otros dispositivos es el 8284A.

\begin{figure}[!htb]
\vskip 5mm
\special{center
         1-2.gif,                
         \the\hsize 78mm}
\vskip 100mm
\caption{Diagrama de bloques del 8088 y 8086} 
\label{Bloques}
\end{figure}

La arquitectura interna del 8086/8088 en diagrama de bloques se puede ver en la %%@
Figura~\ref{Bloques} \cite{Hall}.

El 8086 y el 8088 est\'an divididos en dos partes funcionales independientes, las cu\'ales son %%@
la {\em unidad de interfaz de bus} (BIU) y la {\em unidad de ejecuci\'on} (EU).

Estos procesadores se basan en el dise\~no del 8080/8085. Tienen un conjunto de registros %%@
similares, pero de 16 bits. La BIU alimenta del flujo de instrucciones a la EU, a trav\'es de %%@
una cola precargada de 6 u 8 bytes, de manera que la toma de instrucciones y la ejecuci\'on son %%@
en cierta forma concurrentes. Las instrucciones del 8088/8086 var\'{\i}an de longitud de entre %%@
1 y 4 bytes.

\begin{figure}[!htb]
\vskip 5mm
\special{center
         1-3.gif,                
         \the\hsize 78mm}
\vskip 100mm
\caption{Formato del registro de banderas del Intel 8088.} 
\label{Fig:flags}
\end{figure}

\begin{table}[!hbt]
\begin{center}
\begin{tabular}{|c|c|c|l|} \hline
Bandera & Nombre & Bit & Funci\'on \\ \hline
CF & {\em Carry Flag} & 0 & \parbox{5.5cm}{\vspace{3pt}Vale 1 si una operaci\'on de %%@
adici\'on o substracci\'on produjo acarreo o pr\'estamo.\vspace{3pt}} \\ \hline
PF & {\em Parity Flag} & 2 & \parbox{5.5cm}{\vspace{3pt}Vale 1 si el resultado de una %%@
operaci\'on de datos tiene un n\'umero par de bits con valor 1.\vspace{3pt}} \\ \hline
AF & {\em Auxiliary Carry Flag} & 4 & \parbox{5.5cm}{\vspace{3pt}Se ac\-ti\-va si se ge\-%%@
ne\-r\'o de un a\-ca\-rreo ge\-ne\-ra\-do del cuar\-to bit de un by\-te.\vspace{3pt}} \\ %%@
\hline
ZF & {\em Zero Flag} & 6 & \parbox{5.5cm}{\vspace{3pt}Se activa si el resultado de una %%@
operaci\'on es cero.\vspace{3pt}} \\ \hline
SF & {\em Sign Flag} & 7 & \parbox{5.5cm}{\vspace{3pt}Se activa si el resultado de una %%@
operaci\'on con n\'umeros signados es negativo.\vspace{3pt}} \\ \hline
OF & {\em Overflow Flag} & 11 & \parbox{5.5cm}{\vspace{3pt}Se activa cuando el resultado %%@
de una operaci\'on es mayor que el m\'aximo valor que se puede representar con el n\'umero de %%@
bits del operando destino.\vspace{3pt}} \\ \hline \hline
TF & {\em Trap Flag} & 8 & \parbox{5.5cm}{\vspace{3pt}Cuan\-do se ac\-ti\-va, el mi\-%%@
cro\-pro\-ce\-sa\-dor e\-je\-cu\-ta so\-lo una ins\-truc\-ci\-\'on a la %%@
vez.\vspace{3pt}} \\ \hline
IF & {\em Interrupt Flag} & 9 & \parbox{5.5cm}{\vspace{3pt}Solo cuando este bit esta %%@
activo el 8088 atiende las interrupciones.\vspace{3pt}} \\ \hline
DF & {\em Direction Flag} & 10 & \parbox{5.5cm}{\vspace{3pt}Cuando es activado causa que %%@
el contenido de los registros de \'indice disminuya despu\'es de cada operaci\'on de una cadena %%@
de caracteres.\vspace{3pt}} \\ \hline
\end{tabular}
\caption{Funciones de las banderas del registro de banderas del Intel 8088.}
\label{Tabla:flagfun}
\end{center}
\end{table}

El procesador tiene 4 registros de prop\'osito general (llamados AX, BX, CX, DX), tambi\'en %%@
accesados como registros de 8 bits (AH|AL, BH|BL, CH|CL, DH|DL), y 4 registros \'{\i}ndice de %%@
16 bits (SI, DI, SP ,BP), adem\'as de 4 registros de segmento (DS, CS, ES, SS), que permiten %%@
accesar al CPU hasta 1Mb de memoria. As\'{\i} mismo tiene un {\em registro de banderas\/} %%@
(ver Figura \ref{Fig:flags}), de 16 bits, siendo los 8 inferiores compatibles con el 8085. Solo %%@
9 bits son utilizados, 6 de las cuales indican condiciones producidas por la ejecuci\'on de una %%@
instrucci\'on, y 3 controlan alguna operaci\'on de la unidad de ejecuci\'on. Las banderas que %%@
componen a este registro se presentan en la Tabla \ref{Tabla:flagfun}.

El 8088 tiene puertos de entrada y salida de 64K 8-bits (o 32K 16-bits) y vectores de %%@
interrupci\'on fijos.

El direccionamiento de memoria con los microprocesadores 8088/8086 se realiza sumando el %%@
contenido de los registros de segmento, una vez que se han corrido a la izquierda 4 bits, a la %%@
direcci\'on del desplazamiento \cite{Micro:Hist}.

%----------------------------------------------------------------------------

\subsection{Inicializaci\'on y reinicializaci\'on del procesador.}
\label{Subsection:inicreset}

La inicializaci\'on o arranque del microprocesador 8088 es realizado mediante la puesta a un %%@
nivel l\'ogico ALTO de la terminal RESET. Esta terminal debe estar en ALTO al menos cuatro %%@
ciclos de reloj. El procesador terminar\'a sus operaciones en el borde de subida del RESET. En %%@
la transici\'on de bajada de la se\~nal de RESET el procesador activa una secuencia de reinicio %%@
interno de aproximadamente siete ciclos de reloj de duraci\'on. Despu\'es de esto, el %%@
procesador opera normalmente, comenzando con la instrucci\'on localizada en la direcci\'on %%@
absoluta FFFF0h. La entrada RESET es sincronizada internamente con el reloj del procesador. %%@
Durante la inicializaci\'on, la transici\'on de ALTO a BAJO de la se\~nal RESET debe ocurrir no %%@
antes de 50 $\mu$s despu\'es del encendido, para permitir la completa inicializaci\'on del %%@
8088.

Una Interrupci\'on no Enmascarable, o NMI (ver Subsubsecci\'on~\ref{Subsubsection:NMI}), %%@
ocurrida antes del segundo ciclo de reloj despu\'es del fin del RESET no ser\'a atendida. Si la %%@
NMI es ge\-ne\-ra\-da despu\'es de ese ciclo, y durante la secuencia interna de %%@
inicializaci\'on, el procesador ejecutar\'{\i}a una instrucci\'on antes de ejecutar la %%@
interrupci\'on. Una petici\'on de contenci\'on o toma ({\it hold\/}) inmediatamente despu\'es %%@
del RESET ser\'a atendida antes de la primera carga de instrucci\'on.

Durante el RESET todas las salidas de 3 estados van al tercer estado (OFF) \cite{Intel:Micro}. 

%----------------------------------------------------------------------------

\subsection{Descripci\'on de las terminales del 8088 y 8086.}
\label{Subseccion:terminales}

\begin{figure}[!htb]
\vskip 5mm
\special{center
         8088max.gif,
         \the\hsize 78mm}
\vskip 100mm
\caption{Diagrama de terminales del 8088 en Modo M\'aximo ({\tiny Fuente: OrCad}).} 
\label{8088max}
\end{figure}

El diagrama de terminales del 8088 en modo m\'{\i}nimo se presenta en la %%@
Figura~\ref{8088min}, mientras que en la Figura~\ref{8088max} se presenta el diagrama del %%@
modo m\'aximo. N\'otese que estos diagramas no corresponden a todas las 40 terminales de los %%@
chips, sino solo a los m\'as relevantes para la operaci\'on en cada modo. 

\begin{figure}[!htb]
\vskip 5mm
\special{center
         8088min.gif,
         \the\hsize 78mm}
\vskip 100mm
\caption{Diagrama de terminales del 8088 en Modo M\'{\i}nimo ({\tiny Fuente: OrCad}).} 
\label{8088min}
\end{figure}

La Tabla \ref{Tabla:8088pines} presenta una breve descripci\'on de las terminales del 8088 en %%@
relaci\'on con el modo en que se opere (m\'{\i}nimo o m\'aximo). 

\begin{table}[!htb]
\centering
\begin{tabular}{|c|c|c|} \hline
Terminal No. & Modo M\'{\i}nimo & Modo M\'aximo \\ \hline

9 - 6 & AD7 - AD0 & AD7-AD0 \\ \hline

2-8, 39 & A15-A8 & A15-A8 \\ \hline

35-38 & \parbox{3cm}{\vspace{3pt}A19/S6, A18/S5, A17/S4, A16/S3\vspace{3pt}} & %%@
\parbox{3cm}{\vspace{3pt}A19/S6, A18/S5, A17/S4, A16/S3\vspace{3pt}} \\ \hline

32 & $\overline{\mbox{RD}}$ & $\overline{\mbox{RD}}$ \\ \hline

22 & READY & READY \\ \hline

18 & INTR & INTR \\ \hline

23 & $\overline{\mbox{TEST}}$ & $\overline{\mbox{TEST}}$ \\ \hline

17 & NMI & NMI \\ \hline

21 & RESET & RESET \\ \hline

19 & CLK & CLK \\ \hline

40 & $V_{cc}$ & $V_{cc}$ \\ \hline

1, 20 & GND & GND \\ \hline

33 & $\mbox{MN}/\overline{\mbox{MX}}$ & $\mbox{MN}/\overline{\mbox{MX}}$ \\ %%@
\hline

28 & $\mbox{IO}/\overline{\mbox{M}}$ & $\overline{\mbox{S0}}$ \\ \hline

29 & $\overline{\mbox{WR}}$ & $\overline{\mbox{LOCK}}$ \\ \hline

24 & $\overline{\mbox{INTA}}$ & QS1 \\ \hline

25 & ALE & QS0 \\ \hline

27 & $\mbox{DT}/\overline{\mbox{R}}$ & $\overline{\mbox{S1}}$ \\ \hline

26 & $\overline{\mbox{DEN}}$ & $\overline{\mbox{S2}}$ \\ \hline

31, 30 & HOLD, HOLDA & $\overline{\mbox{RQ}}/\overline{\mbox{GT1}}$,  %%@
$\overline{\mbox{RQ}}/\overline{\mbox{GT0}}$ \\ \hline

34 & $\overline{\mbox{SS0}}$ & - \\ \hline
\end{tabular}
\caption{Terminales del Intel 8088.}
\label{Tabla:8088pines}
\end{table}

Los requerimientos para soportar sistemas 8088 m\'{\i}nimo y m\'aximo son diferentes, por lo %%@
que no se pueden llevar a cabo con 40 terminales. Debido a esto, el 8088 est\'a equipado con una %%@
terminal de selecci\'on ($MN/\overline{MX}$, terminal n\'umero 33) la cual define la %%@
configuraci\'on del sistema. La definici\'on de determinado subconjunto de cambios en las %%@
terminales depende del estado de la terminal $MN/\overline{MX}$. Cuando $MN/\overline{MX}$ %%@
est\'a a GND (tierra), el 8088 define las terminales 24 a 31 y 34 en modo m\'aximo. En este %%@
modo se requiere que las se\~nales de control del bus sean generadas por el circuito integrado %%@
8288. Cuando la terminal $MN/\overline{MX}$ esta a $V_{cc}$, el 8088 genera las se\~nales de %%@
control de bus por s\'{\i} mismo de las terminales 24 a 31 y 34.

\begin{figure}[!hbt]
\vskip 15mm
\special{center
         1-6.gif,                
         \the\hsize 78mm}
\vskip 110mm
\caption{Configuraci\'on de bus multiplexado.} 
\label{multiplexado}
\end{figure}

Tanto el modo m\'{\i}nimo como el modo m\'aximo del 8088 pueden ser usados con bus %%@
multiplexado o demultiplexado. La configuraci\'on de bus multiplexado %%@
(Figura~\ref{multiplexado}) brinda al usuario con un sistema de chips m\'{\i}nimo, sin perder %%@
poder de procesamiento \cite{Intel:Micro}.

El modo demultiplexado requiere un {\em latch}\footnote{Circuito que permite almacenar bits, %%@
usualmente tambi\'en se les llama {\it registros\/}.} para direccionamiento de 64K, o dos %%@
{\em latches} para direccionar 1Mb completo. Se puede usar un tercer {\em latch} para %%@
establecer un {\em buffer}\footnote{Memoria intermedia.} si la carga del bus de direcciones %%@
lo requiere; un {\em transceiver}\footnote{Dispositivo que permite comunicar o asilar dos %%@
l\'{\i}neas de transmisi\'on de datos, o buses.} puede ser usado tambi\'en en este caso (ver %%@
Figura~\ref{demultiplexado}). Las se\~nales $\overline{DEN}$ y $DT/\overline{R}$ del %%@
microprocesador son utilizadas para controlar el {\em transceiver} y la se\~nal $ALE$ es %%@
utilizada para habilitar el {\em latch\/} para que pueda retener las direcciones %%@
\cite{Intel:Micro}.

\begin{figure}[!hbt]
\vskip 15mm
\special{center
         1-7.gif,                
         \the\hsize 78mm}
\vskip 115mm
\caption{Configuraci\'on de bus demultiplexado (arriba) y de Sistema con buffers completos %%@
usando el controlador de bus (abajo).}
\label{demultiplexado}
\end{figure}

El modo m\'aximo emplea, como se dice antes, el controlador de bus 8288. El 8288  decodifica las %%@
l\'{\i}neas de estado $\overline{S0}$, $\overline{S1}$ y $\overline{S2}$, y genera %%@
todas las se\~nales de control de bus. El 8288 proporciona un mejor control del bus %%@
\cite{Intel:Micro} y adem\'as libera terminales del 8088 para aumentar las capacidades de %%@
\'este, tales como permitir la conexi\'on de coprocesadores (como el 8087) al bus local, y %%@
configuraciones de bus remoto. 

El 8088 requiere que el reloj CLK sea controlado por un generador de reloj externo controlado %%@
por un cristal (el generador de reloj y manejador 8284) para sincronizar sus operaciones %%@
internas. Las frecuencias del reloj para el 8088 pueden ser de 5 Mhz para el 8088 y de 8 Mhz %%@
para el 8088-2.

%----------------------------------------------------------------------------

\subsubsection{Modo M\'{\i}nimo.}
\label{Subsubsection:modominimo}

En modo m\'{\i}nimo (ver Figura~\ref{modomin}), las terminales al bus est\'an multiplexados, %%@
para poder colocar al 8088 en un circuito integrado de 40 terminales. Utiliza las terminales del %%@
n\'umero 9 al 16 para el bus de datos (8 bits) y direcciones y del 2 al 8 y del 35 al 39 para el %%@
bus de direcciones \'unicamente. Esto da los 8 bits de datos y 20 de direcciones. 

Cabe aclarar que el 8088 es operado en modo m\'{\i}nimo cuando es el \'unico microprocesador %%@
en los buses del sistema.

%----------------------------------------------------------------------------

\subsubsection{Modo M\'aximo.}
\label{Subsubseccion:modomaximo}

El modo m\'aximo es activado colocando la terminal $MN/\overline{MX}$ en bajo, como se %%@
menciona en la Subsecci\'on~\ref{Subseccion:terminales}. 

%----------------------------------------------------------------------------
