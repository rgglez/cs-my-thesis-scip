\markboth{Conclusiones y Perspectivas.}{}
\vspace*{20pt}
\begin{flushleft}
{\LARGE\bf Conclusiones y Perspectivas.}
\end{flushleft}
\vskip 50pt
\label{Chapter:conclusiones}

%----------------------------------------------------------------------------

\subsubsection{Conclusiones.}
\label{Subsubsection:conclusiones}

El ``Sistema de Coprocesamiento de Informaci\'on para PC'', resultado de este trabajo, permite %%@
efectuar la comunicaci\'on entre dos procesadores, y la ejecuci\'on de un programa en la tarjeta %%@
de expansi\'on, enviando los resultados al procesador de la computadora. El uso del Controlador %%@
Programable de Perif\'ericos 8255 permite que la tarjeta de expansi\'on vea a la PC como a un %%@
perif\'erico.

Se desarroll\'o software b\'asico para demostrar la comunicaci\'on bidireccional entre la %%@
tarjeta SCIP y la computadora anfitriona, y al mismo tiempo la ejecuci\'on de programas distintos %%@
en ambos procesadores.

\subsubsection{Perspectivas}
\label{Subsubsection:perspectivas}

El dise\~no presentado puede considerarse b\'asico, y podr\'a ser personalizado por un usuario, %%@
con el uso de alg\'un tipo de interfaz (serial, paralela  u otra), o el uso de l\'ogica %%@
adicional, para agregarle elementos externos, y expandir sus aplicaciones a \'areas como %%@
adquisici\'on de datos, o procesamiento de sonido o gr\'aficos.

Dadas las limitaciones de los recursos disponibles para el desarrollo del sistema, la %%@
circuiter\'{\i}a utilizada en el mismo podr\'{\i}a considerarse obsoleta, habiendo ya %%@
procesadores mas modernos o especializados en ciertas tareas.  
Como mejoras al sistema se puede contemplar la substituci\'on de los circuitos integrados %%@
utilizados (el procesador y toda su l\'ogica de soporte) por otros m\'as modernos, as\'{\i} %%@
como el uso de memoria de otro tipo, que permita expandir la capacidad de almacenamiento del %%@
sistema. Sin embargo, el 8088 era la \'unica opci\'on disponible que permitia simplificar la %%@
l\'ogica del sistema, ya que si se haciera uso de otro tipo de procesadores (de la familia Intel) %%@
se requerir\'{\i}a el uso de otros dispositivos y t\'ecnicas.

As\'{\i} mismo, el software para una aplicaci\'on espec\'{\i}fica del sistema debe ser %%@
desarrollado. En el sistema actual, este software propiamente no existe, y solo se contemplaron %%@
peque\~nos programas de prueba para efectuar la comunicaci\'on bidireccional entre la %%@
computadora anfitriona y SCIP, y un software b\'asico grabado en una memoria EPROM de la tarjeta, %%@
para programar los dispositivos que as\'{\i} lo requieren, y efectuar la comunicaci\'on %%@
mencionada.

%----------------------------------------------------------------------------------------
