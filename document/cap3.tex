\chapter[Dise\~no y Descripci\'on del Sistema.]{Dise\~no y Descripci\'on del Sistema.}
\label{Capitulo:diseno}

%----------------------------------------------------------------------------

\fancyhead[RE]{\slshape SCIP - Cap. \thechapter~ - Dise\~no y Descripci\'on del %%@
Sistema.}
\fancyhead[LO]{\slshape Rodolfo Gonz\'alez Gonz\'alez.}
\fancyhead[LE,RO]{\thepage}

%----------------------------------------------------------------------------

\section{Introducci\'on.}
\label{Seccion:IntroCap3}

Este cap\'{\i}tulo presenta la descripci\'on del dise\~no de la tarjeta SCIP. La tarjeta SCIP %%@
fue ensamblada usando bases para alambramiento\footnote{Estas bases para alambramiento o wire-%%@
wrap proporcionan la ventaja de que no es necesario soldar postes extra (como ser\'{\i}a el %%@
caso si se usaran sockets para C.I.'s), ahorran espacio y son f\'aciles de alambrar o %%@
desalambrar los circuitos ensamblados con ellas. Sin embargo elevan el costo del dise\~no, ya %%@
que son de 3 a 5 veces mas caras que un socket normal para integrados.} sobre una tarjeta para %%@
prototipos de 16 bits JDR-PR10 de JDR Microdevices, la cual se monta, para prop\'ositos de %%@
prueba y mediciones, sobre una tarjeta de extensi\'on EXT-80286. La tarjeta JDR-PR10 tiene %%@
integrado un dise\~no para la decodificaci\'on y generaci\'on de las l\'{\i}neas de %%@
selecci\'on de la tarjeta, y ``buffers'' para los buses de direcci\'on, datos y otras %%@
l\'{\i}neas provenientes de conectores de expansi\'on de la PC. Sin embargo, dado que el %%@
dise\~no de JDR utiliza PAL's (Arreglos L\'ogicos Programables) en la l\'ogica de %%@
decodificaci\'on, los cuales no se pudieron obtener, se realiz\'o dicha l\'ogica %%@
independientemente.

El dise\~no se divide en 3 m\'odulos principales:

\begin{enumerate}
\item M\'odulo de procesamiento.
\item M\'odulo de memoria.
\item M\'odulo de comunicaci\'on entre la tarjeta SCIP y la computadora.
\end{enumerate}

El diagrama por bloques se presenta en la Figura \ref{Figura:SCIPbloques}.

\begin{figure}[!htb]\centering
%TexCad Options
%\grade{\off}
%\emlines{\off}
%\beziermacro{\on}
%\reduce{\on}
%\snapping{\off}
%\quality{2.00}
%\graddiff{0.01}
%\snapasp{1}
%\zoom{1.00}
\unitlength 1mm
\linethickness{0.4pt}
\begin{picture}(150.33,70.00)
\put(4.67,10.00){\framebox(145.67,60.00)[cc]{ }}
\put(96.33,3.00){\framebox(52.67,7.00)[cc]{ }}
\put(71.67,3.00){\framebox(21.00,7.00)[cc]{ }}
\put(70.67,12.00){\framebox(73.33,25.33)[cc]{1}}
\put(7.33,42.00){\framebox(40.00,25.67)[cc]{2}}
\put(50.00,42.00){\framebox(22.33,25.67)[cc]{3}}
\put(7.33,12.00){\framebox(12.00,26.67)[cc]{4}}
\put(22.00,12.00){\framebox(25.33,26.67)[cc]{5}}
\put(50.00,12.00){\framebox(18.67,26.33)[cc]{6}}
\put(75.67,42.00){\framebox(12.33,25.67)[cc]{ROM1}}
\put(90.67,42.00){\framebox(13.00,25.67)[cc]{ROM2}}
\put(106.33,42.00){\framebox(37.67,25.67)[cc]{7}}
\end{picture}
\caption{Diagrama en bloques de SCIP: (1) Secci\'on preconstruida de decodificaci\'on, (2) %%@
M\'odulo de Procesamiento y Control de BUS, (3) M\'o\-du\-lo de de\-co\-di\-fi\-ca\-ci\-%%@
\'on in\-ter\-na y bu\-ffers, (4) M\'odulo de temporizaci\'on, (5) M\'odulo de %%@
comunicaci\'on, (6) M\'odulo de l\'ogica de selecci\'on, (7) M\'odulo de memorias RAM, (ROM1) %%@
Memoria EPROM 1, (ROM2) Memoria EPROM 2.} 
\label{Figura:SCIPbloques}
\end{figure}

En \'este cap\'{\i}tulo se hace referencia a las hojas de diagramas presentadas en el %%@
Ap\'endice \ref{Apendice:diagramas}.

%----------------------------------------------------------------------------

\section{M\'odulo de procesamiento.}
\label{Section:modproc}

La funci\'on del m\'odulo de procesamiento es la de ejecutar el programa de usuario que reside %%@
en una memoria EPROM dentro de la tarjeta SCIP. 

Este m\'odulo consta de las siguientes partes principales:

\begin{itemize}
\item Un microprocesador Intel 8088.
\item Un coprocesador matem\'atico Intel 8087.
\item Un temporizador Intel 8284.
\item Un controlador de interrupciones programable 8259.
\item Un controlador de bus 8288.
\item Latches para las l\'{\i}neas de datos/direcciones.
\item Circuito para generar el RESET de este m\'odulo.
\end{itemize}

Esta parte del dise\~no se puede ver en las hojas de diagrama n\'umeros 4 (``M\'odulo de %%@
generaci\'on de l\'{\i}neas de control/IRQ's'') y 1 (``M\'odulo de Procesamiento'')  del %%@
Ap\'endice \ref{Apendice:diagramas}.

Como se ve en la hoja 1, el microprocesador Intel 8088 esta conectado en modo m\'aximo (el cual %%@
se define en la Subsubsecci\'on \ref{Subsubseccion:modomaximo}, la Subsecci\'on %%@
\ref{Subseccion:arquitectura} y la Subsecci\'on \ref{Subseccion:terminales}), para poder %%@
soportar la utilizaci\'on del coprocesador matem\'atico Intel 8087. La conexi\'on de los buses %%@
de datos y direcciones de ambos se realiza en paralelo (ver Subsecci\'on %%@
\ref{Subsection:confop8087}). As\'{\i} mismo la conexi\'on de las l\'{\i}neas de estado S1-%%@
S3 de ambos circuitos se realiza en paralelo, para poder decodificar el estado del bus en un %%@
momento dado. Estas l\'{\i}neas de estado se conectan a las entradas correspondientes del 8288 %%@
(terminales 3, 18 y 19), el cu\'al se encarga de la generaci\'on de las l\'{\i}neas de control %%@
del bus ($\overline{\mbox{MRD}}$, $\overline{\mbox{MWT}}$, %%@
$\overline{\mbox{AWM}}$, $\overline{\mbox{IOR}}$, $\overline{\mbox{IOW}}$, %%@
$\overline{\mbox{AIOW}}$, DT/$\overline{R}$,  DEN, MC/$\overline{\mbox{PD}}$, y %%@
ALE). Las l\'{\i}neas de control que son usadas en SCIP son ALE, DT/$\overline{R}$,  %%@
$\overline{\mbox{MRD}}$, y $\overline{\mbox{MWT}}$ para la selecci\'on de memoria %%@
local, y $\overline{\mbox{IOR}}$ y $\overline{\mbox{IOW}}$ para la selecci\'on de %%@
puertos. As\'{\i} mismo, se utilizan para controlar el latch 74LS373 y los transceivers 74LS245 %%@
(U4 y U5/U6 respectivamente, en la hoja 1 del circuito), para mantener el tr\'afico en el bus de %%@
datos y de direcciones de acuerdo a las especificaciones dadas anteriormente (ver %%@
Secci\'on~\ref{Seccion:bus}).

La temporizaci\'on del sistema se realiza mediante un generador de reloj 8284 (U1 en la hoja 4), %%@
con un cristal de 14.31818 Mhz, frecuencia que es dividida por 3 para la adecuada operaci\'on %%@
del 8088 , 8087 y del 8288, la cu\'al es de 4.74 $\approx$ 5 Mhz. Este mismo circuito %%@
integrado sirve para generar la se\~nal de RESET para el microprocesador 8088 (ver Subsecci\'on %%@
\ref{Subsection:inicreset}). El circuito utilizado para la generaci\'on de la se\~nal de RESET %%@
consta de una resistencia de 640 K$\Omega$, un capacitor de 360 $\mu$F y un interruptor de %%@
bot\'on. Al presionarse el bot\'on se pone en BAJO la terminal RES del 8284, con lo que \'este %%@
genera el pulso de RESET para el 8088 (y en consecuencia para el 8087, 8255 y el 8288). Cabe %%@
se\~nalar que la duraci\'on de la se\~nal de RESET es de importancia, debido a que debe de %%@
estar presente al menos durante 4 ciclos de reloj para que el procesador pueda iniciar su %%@
operaci\'on normalmente \cite{Intel:Micro}. 

En el dise\~no realizado es necesario el uso de las l\'{\i}neas AEN2-AEN1 y RDY2-RDY1 ya que %%@
estas se necesitan para configuraciones con m\'utiples maestros de bus; debido a esto, AEN1-AEN2 %%@
se colocan en BAJO y RDY1-RDY2 se colocan en ALTO. La terminal F/$\overline{\mbox{C}}$ se %%@
coloca a BAJO para habilitar la generaci\'on de la se\~nal del reloj por medio del cristal.

La terminal $\overline{\mbox{TEST}}$ del 8088 se conecta a la terminal BUSY del 8087, para %%@
hacer posible la sincronizaci\'on de ambos dispositivos, como se se\~nala en la Subsecci\'on %%@
\ref{Subsection:confop8087}. Con el mismo prop\'osito de sincronizaci\'on se conecta la %%@
terminal RQ/$\overline{\mbox{GT1}}$ (no. 30) del 8088 con la terminal %%@
RQ/$\overline{\mbox{GT0}}$ (no. 31) del 8087; en este caso, estas l\'{\i}neas controlan %%@
las peticiones o concesiones del bus, para que en caso de que el 8087 necesite tomar el control %%@
del bus, se coordine con el 8088.

Como se menciona arriba, el control del bus del sistema interno de la tarjeta, al estar el 8088 %%@
conectado en modo m\'aximo, se lleva a cabo con un Controlador de Bus 8288. Las l\'{\i}neas de %%@
estado del 8088 (S0-S2) indican el modo de operaci\'on del bus (ver Subsecci\'on %%@
\ref{Subseccion:terminales}). Estas l\'{\i}neas se conectan a las l\'{\i}neas S0-S2 del %%@
8288, el cual, a trav\'es de sus l\'{\i}neas de control determina la operaci\'on actualmente %%@
llevada a cabo por el bus.

El manejo de las excepciones que puedan ocurrir en el 8087 (ver Subsecci\'on %%@
\ref{Subsection:excep8087}) se lleva a cabo mediante un PIC (Controlador de Interrupciones %%@
Programable) 8259 (U2 en la hoja 4 del diagrama). La terminal INT del 8087 se conecta a la %%@
terminal IR0 del 8259. Esto se hace con el fin de generar una petici\'on de interrupci\'on. %%@
Esta se mantiene activa hasta que el 8288 genera una se\~nal de aceptaci\'on de interrupci\'on %%@
($\overline{\mbox{INTA}}$). La terminal INTR del 8088 se conecta al 8259 para para en caso %%@
de producirse una petici\'on v\'alida de interrupci\'on, el 8259 interrumpa a la CPU 8088. Las %%@
l\'{\i}neas del bus de datos de la tarjeta (D0-D7) se conectan a las terminales del bus de %%@
datos del 8259 (D0-D7), las cuales son de tres estados, para que por ellas se transmita la %%@
informaci\'on de control, estado y vector de interrupci\'on. Este bus es bidireccional, y %%@
proviene del transceiver 74LS245 (U5 en la hoja 1 del diagrama). En el caso de SCIP, el 8259 %%@
trabaja como \'unico PIC del sistema, por lo que las l\'{\i}neas CAS0-CAS1 se mantienen %%@
desconectadas, y la terminal 16 (SP/$\overline{\mbox{EN}}$) se pone a un nivel ALTO. 

Durante la inicializaci\'on del sistema, el software del BIOS de SCIP realiza la programaci\'on %%@
del 8259, mediante el c\'odigo que se presenta a continuaci\'on:

\vspace{12pt}

\begin{verbatim}
MOV  DX,3C0h       ; Primer puerto  (ICW 1)
MOV  AL,17h        ; Modo de disparo por borde, intervalo de 4, 
                   ; sencillo, se requiere IC4 
OUT  DX,AL         ; 00010111

MOV  DX,3C1h       ; Segundo puerto (ICW 2)
MOV  AL,0h         ; Inicio del vector de interrupciones
OUT  DX,AL         ; 00000000

MOV  DX,3C1h       ; Segundo puerto (ICW 3)
MOV  AL,0h         ; Esta palabra es ignorada, ya que 
                   ; solo hay un PIC
OUT  DX,AL         ; 00000000

MOV  DX,3C1h       ; Segundo puerto (ICW 4)
MOV  AL,5h         ; Modo no especial completamente anidado
                   ; modo sin buffers, EOI normal, 
                   ; modo para 8086/8088 
OUT  DX,AL         ; 00000101
\end{verbatim}

\vspace{12pt}

El programa escribe a los puertos 3C0h y 3C1h en donde se encuentra el 8259 (ver Subsecci\'on %%@
\ref{Subseccion:internalports}), y mediante cuatro Palabras de Control para Inicializaci\'on %%@
(ICW1-ICW4) lo programa:

\begin{enumerate}
\item La primera palabra programa el modo de disparo por borde, un intervalo de 4 bytes entre %%@
los vectores de interrupci\'on, \'unico PIC del sistema, y palabra de control 4 necesaria.
\item La segunda palabra de control indica al 8259 que el vector de interrupciones inicia en la %%@
direcci\'on 0 absoluta.
\item La tercera palabra de control es ignorada, puesto que el 8259 es el \'unico PIC del %%@
sistema.
\item La cuarta palabra indica al PIC que trabaje en modo no especial completamente anidado, en %%@
modo ``sin buffers'', con EOI (Fin De In\-te\-rrup\-ci\-\'on) normal, y en modo 8088/8086.
\end{enumerate}

%----------------------------------------------------------------------------

\section{M\'odulo de memoria, puertos y decodificaci\'on interna.}
\label{Section:modmem}

%----------------------------------------------------------------------------

\subsection{Memoria interna de SCIP.}
\label{Subseccion:memo}

La funci\'on del m\'odulo de memoria es el almacenar las rutinas de inicializaci\'on de la %%@
tarjeta, el programa de usuario que residir\'a en la tarjeta, y proporcionar un \'area de %%@
trabajo en RAM para tal programa.

Este m\'odulo se presenta en las hojas de diagrama 5 y 6 del Ap\'endice %%@
\ref{Apendice:diagramas}, y consta de:

\begin{itemize}
\item Dos EPROM\footnote{Memoria de Solo Lectura El\'ectricamente Programables y Borrables} %%@
NEC 27512 de 64Kb $\times$ 8 bits cada una.
\item Cuatro RAM\footnote{Memoria de Acceso Aleatorio} est\'aticas\footnote{La raz\'on de %%@
la utilizaci\'on de memorias RAM est\'aticas en lugar de usar memorias din\'amicas o es que %%@
para estas no se requiere l\'ogica adicional para el refresco de las memorias, y dado que el %%@
espacio en la tarjeta para prototipos JDR-PR10 usada es reducido, se prefiri\'o utilizar %%@
memorias est\'aticas. Sin embargo, el costo de estas elev\'o el costo global del sistema.} NEC %%@
PD43256A de 32Kb $\times$ 8 bits cada una.
\item Decodificadores/demultiplexores 2 a 4 74LS139.
\item Compuertas OR 74LS32.
\item Decodificadores 1 de 8 74LS138.
\item Compuertas AND 74LS08.
\item Compuerta AND 74LS10.
\item Compuerta NOT 74LS04.
\end{itemize}

En la tarjeta SCIP se tienen 128 Kb de memoria ROM y 128 Kb de memoria RAM. 64 Kb de memoria ROM %%@
son destinados a las rutinas de inicializaci\'on de la tarjeta, y el circuito integrado que las %%@
contiene permanece en ella siempre (bloque 6, ver Tabla \ref{Tabla:decodifmem}). Los otros 64 %%@
Kb de ROM estan destinados al programa de aplicaci\'on espec\'{\i}fico del usuario que reside %%@
en la tarjeta, y el circuito integrado puede ser intercambiado para colocar otro programa %%@
dependiendo de las necesidades del usuario (bloque 5, ver Tabla \ref{Tabla:decodifmem}). El %%@
circuito integrado de memoria EPROM que permanece en la tarjeta ocupa el rango de direcciones %%@
m\'as alto (F0000-FFFFF hexadecimal, es decir, el bloque de memoria 6, ver Tabla %%@
\ref{Tabla:decodifmem}) ya que dadas la caracter\'{\i}stica del reset e inicializaci\'on del %%@
8088, este salta a la direcci\'on FFFF0 hexadecimal (ver Subsecci\'on %%@
\ref{Subsection:inicreset}). Por tanto, el bloque 5 (ver Tabla \ref{Tabla:decodifmem}) de %%@
memoria es la EPROM intercambiable que contendr\'a el programa del usuario.

El mapa de la memoria interna de SCIP se presenta en la Figura \ref{Fig:mapamem}.

\begin{figure}[!htb]
\vskip 5mm
\special{center
         mapamem.gif,
         \the\hsize 78mm}
\vskip 78mm
\caption{Mapa de memoria de la tarjeta SCIP.}
\label{Fig:mapamem}
\end{figure}

La Tabla \ref{Tabla:decodifmem} presenta los rangos de direcciones utilizados para la memoria %%@
de la tarjeta SCIP y la respectiva direcci\'on de 20 bits como se presenta en el bus de %%@
direcciones del sistema. En binario se presenta solo la direcci\'on de inicio de cada bloque de %%@
memoria.

\vspace{12pt}

\begin{table}[!htb]
\centering
\begin{tabular}{|c|c|c|c|c|c|c|c|} \hline
{\tiny $A_{19} - A_{16}$} & {\tiny $A_{15} - A_{12}$} & {\tiny $A_{11} - A_{8}$} & %%@
{\tiny $A_{7} - A_{4}$} & {\tiny $A_{3} - A_{0}$} & {\tiny Inicio} & {\tiny Fin} & %%@
{\tiny Bloque} \\ \hline \hline
{\footnotesize 0000} & {\footnotesize 0000} & {\footnotesize 0000} & %%@
{\footnotesize 0000} & {\footnotesize 0000} & {\footnotesize 00000} & %%@
{\footnotesize 07FFF} & {\footnotesize 1} \\ \hline
{\footnotesize 0000} & {\footnotesize 1000} & {\footnotesize 0000} & %%@
{\footnotesize 0000} & {\footnotesize 0000} & {\footnotesize 08000} & %%@
{\footnotesize 0FFFF} & {\footnotesize 2} \\ \hline
{\footnotesize 0001} & {\footnotesize 0000} & {\footnotesize 0000} & %%@
{\footnotesize 0000} & {\footnotesize 0000} & {\footnotesize 10000} & %%@
{\footnotesize 17FFF} & {\footnotesize 3} \\ \hline
{\footnotesize 0001} & {\footnotesize 1000} & {\footnotesize 0000} & %%@
{\footnotesize 0000} & {\footnotesize  0000} & {\footnotesize 18000} & %%@
{\footnotesize 1FFFF} & {\footnotesize 4} \\ \hline
{\footnotesize 1110} & {\footnotesize 0000} & {\footnotesize 0000} & %%@
{\footnotesize 0000} & {\footnotesize 0000} & {\footnotesize E0000} & %%@
{\footnotesize EFFFF} & {\footnotesize 5} \\ \hline
{\footnotesize 1111} & {\footnotesize 0000} & {\footnotesize 0000} & %%@
{\footnotesize 0000} & {\footnotesize 0000} & {\footnotesize F0000} & %%@
{\footnotesize FFFFF} & {\footnotesize 6} \\ \hline
\end{tabular}
\caption{Rangos de direcciones de los bloques de memoria de SCIP.}
\label{Tabla:decodifmem}
\end{table}

\vspace{12pt}

La decodificaci\'on de las direcciones para la selecci\'on de los circuitos integrados de %%@
memoria se lleva a cabo usando decodificadores 74LS139\footnote{El 74LS139 es un %%@
decodificador/demultiplexor dual 1 de 4.}. 

Para la selecci\'on de los circuitos de memoria RAM se conectan las l\'{\i}neas A15 y A16 del %%@
bus de direcciones de SCIP a las entradas A0 y A1, respectivamente, de un 74LS139, como se %%@
muestra en la Tabla \ref{Tabla:decodif1}. La entrada de habilitaci\'on %%@
($\overline{\mbox{E}}$ de dicho 74LS139 se conecta a la l\'{\i}nea A19, ya que en el rango %%@
de direcciones de memoria que ocupa la memoria RAM la l\'{\i}nea A19 del bus de direcciones de %%@
SCIP siempre esta en BAJO (ver Tabla \ref{Tabla:decodifmem}).

\begin{table}[!htb]
\centering
\begin{tabular}{|c|c|c|} \hline
A15 $\rightarrow$ A0 & A16 $\rightarrow$ A1 & L\'{\i}nea habilitada en el 74LS139 \\ %%@
\hline
0 & 0 & $\overline{\mbox{O0}}$ \\
1 & 0 & $\overline{\mbox{O1}}$ \\
0 & 1 & $\overline{\mbox{O2}}$ \\
1 & 1 & $\overline{\mbox{O3}}$ \\ \hline
\end{tabular}
\caption{Entradas del 74LS139 para la selecci\'on de las memorias RAM}
\label{Tabla:decodif1}
\end{table}

Para seleccionar los circuitos de memoria EPROM se utilizan las l\'{\i}neas A15 y A16 del bus %%@
de direcciones de SCIP como entrada para A0 y A1 respectivamente en un 74LS139, como se muestra %%@
en la Tabla \ref{Tabla:decodif2}. La entrada de habilitaci\'on del 74LS139 se conecta a la %%@
salida de una compuerta 74LS04 la cual invierte la l\'{\i}nea A19 del bus de SCIP ya que esta %%@
siempre esta en ALTO en el rango de direcciones que comprende las dos memorias EPROM (ver Tabla %%@
\ref{Tabla:decodifmem}).

\begin{table}[!htb]
\centering
\begin{tabular}{|c|c|c|} \hline
A15 $\rightarrow$ A0 & A16 $\rightarrow$ A1 & L\'{\i}nea habilitada en el 74LS139 \\ %%@
\hline
0 & 0 & $\overline{\mbox{O0}}$ \\
0 & 1 & $\overline{\mbox{O2}}$  \\ \hline
\end{tabular}
\caption{Entradas del 74LS139 para la selecci\'on de las memorias EPROM}
\label{Tabla:decodif2}
\end{table}

%----------------------------------------------------------------------------

\subsection{Puertos internos.}
\label{Subseccion:internalports}

La l\'ogica de decodificaci\'on presentada en la hoja de diagramas n\'umero 5 permite %%@
seleccionar hasta 8 puertos dentro del sistema SCIP\footnote{Este n\'umero se podr\'{\i}a %%@
aumentar con l\'ogica adicional, usando A3=1 y otro 74LS138, por ejemplo.}. Para esta parte del %%@
circuito se utilizan los decodificadores U5 y U6 (ambos 74LS138). En U5 se genera la se\~nal Y7 %%@
en caso de que la direcci\'on presente en las l\'{\i}neas A4 - A9 del bus de direcciones %%@
interno sea 3C{\it x\/}. Y7 entonces habilita al decodificador U6 el cu\'al decodifica las %%@
l\'{\i}neas A0, A1 y A2 para la selecci\'on de puertos (A3=0).

Los puertos internos de SCIP decodificados por U6 son 2 para el 8259 y 4 para el 8255, como se %%@
muestra en la Tabla \ref{Tabla:puertointernos}.

% Table created by WinTeX 95: 3 Columns x 7 Rows.
\begin{table}[!htb]\centering
\begin{tabular}{|l|l|l|}
%Row: 1
\cline{1-3}
\vbox to1.70ex{\vspace{1pt}\vfil\hbox to13.20ex{\hfil A9 - A0\hfil}\vfil} & 
\vbox to1.70ex{\vspace{1pt}\vfil\hbox to13.20ex{\hfil Hex.\hfil}\vfil} & 
\vbox to1.70ex{\vspace{1pt}\vfil\hbox to12.80ex{\hfil Disp.\hfil}\vfil} \\

%Row: 2
\cline{1-3}
\vbox to1.70ex{\vspace{1pt}\vfil\hbox to13.20ex{\hfil 1111000000\hfil}\vfil} & 
\vbox to1.70ex{\vspace{1pt}\vfil\hbox to13.20ex{\hfil 3C0\hfil}\vfil} & 
\vbox to1.70ex{\vspace{1pt}\vfil\hbox to12.80ex{\hfil 8259\hfil}\vfil} \\

%Row: 3
\cline{1-3}
\vbox to1.70ex{\vspace{1pt}\vfil\hbox to13.20ex{\hfil 1111000001\hfil}\vfil} & 
\vbox to1.70ex{\vspace{1pt}\vfil\hbox to13.20ex{\hfil 3C1\hfil}\vfil} & 
\vbox to1.70ex{\vspace{1pt}\vfil\hbox to12.80ex{\hfil 8259\hfil}\vfil} \\

%Row: 4
\cline{1-3}
\vbox to1.70ex{\vspace{1pt}\vfil\hbox to13.20ex{\hfil 1111000100\hfil}\vfil} & 
\vbox to1.70ex{\vspace{1pt}\vfil\hbox to13.20ex{\hfil 3C4\hfil}\vfil} & 
\vbox to1.70ex{\vspace{1pt}\vfil\hbox to12.80ex{\hfil 8255\hfil}\vfil} \\

%Row: 5
\cline{1-3}
\vbox to1.70ex{\vspace{1pt}\vfil\hbox to13.20ex{\hfil 1111000101\hfil}\vfil} & 
\vbox to1.70ex{\vspace{1pt}\vfil\hbox to13.20ex{\hfil 3C5\hfil}\vfil} & 
\vbox to1.70ex{\vspace{1pt}\vfil\hbox to12.80ex{\hfil 8255\hfil}\vfil} \\

%Row: 6
\cline{1-3}
\vbox to1.70ex{\vspace{1pt}\vfil\hbox to13.20ex{\hfil 1111000110\hfil}\vfil} & 
\vbox to1.70ex{\vspace{1pt}\vfil\hbox to13.20ex{\hfil 3C6\hfil}\vfil} & 
\vbox to1.70ex{\vspace{1pt}\vfil\hbox to12.80ex{\hfil 8255\hfil}\vfil} \\

%Row: 7
\cline{1-3}
\vbox to1.70ex{\vspace{1pt}\vfil\hbox to13.20ex{\hfil 1111000111\hfil}\vfil} & 
\vbox to1.70ex{\vspace{1pt}\vfil\hbox to13.20ex{\hfil 3C7\hfil}\vfil} & 
\vbox to1.70ex{\vspace{1pt}\vfil\hbox to12.80ex{\hfil 8255\hfil}\vfil} \\

\cline{1-3}
\end{tabular}
\caption{Direcciones de los puertos para los dispositivos PIC y PPI.}
\label{Tabla:puertointernos}
\end{table}

Estos puertos se utilizan para programar y acceder tanto al PIC 8259 como al PPI 8255. Es %%@
necesario contar con tantos puertos debido a que ambos dispositivos lo requieren as\'{\i} (ver %%@
Ap\'endice \ref{Apendice:chips}).

%----------------------------------------------------------------------------

\section{M\'odulo de comunicaci\'on.}
\label{Section:decodifbus}

La funci\'on de este m\'odulo es realizar la comunicaci\'on entre la tarjeta SCIP y la %%@
computadora en ambos sentidos.

Este m\'odulo se divide en dos partes:

\begin{enumerate}
\item Decodificaci\'on de l\'{\i}neas de direcci\'on para selecci\'on de la tarjeta.
\item Puertos de comunicaci\'on PC $\Longleftrightarrow$ SCIP.
\end{enumerate}

%----------------------------------------------------------------------------

\subsection{T\'ecnicas de decodificaci\'on para selecci\'on de dispositivos.}
\label{Subsection:tecnicas}

Para la decodificaci\'on y sensado de muchos dispositivos perif\'ericos y de entrada y salida %%@
en la PC se utilizan los puertos (ver Secci\'on \ref{Section:puertos}). Para habilitar un %%@
dispositivo colocado en una ranura de expansi\'on se necesita decodificar de alguna manera las %%@
l\'{\i}neas de direccion del bus de la PC. Esto se puede llenar a cabo de las siguientes %%@
formas:

\begin{enumerate}
\item Mapeo de memoria.
\item Puertos de entrada/salida.
\end{enumerate}

El primer m\'etodo consiste en ver cada perif\'erico como una direcci\'on \'unica en la %%@
memoria de la computadora, siendo posible con esto el uso de todas las instrucciones que sirven %%@
para accesar a memoria.

El segundo m\'etodo consiste en ver al perif\'erico como un dispositivo de entrada/salida, %%@
independiente de la memoria. Con esto se limita el acceso al dispositivo a las instrucciones IN %%@
y OUT.

Las formas para seleccionar un dispositivo son:

\begin{enumerate}
\item L\'ogica aleatoria.
\item Decodificadores.
\item Comparadores.
\item Memorias ROM programables (PROM's) de mapeo.
\end{enumerate}

El m\'etodo de decodificaci\'on consiste en utilizar un circuito integrado decodificador tal %%@
como el 74LS138\footnote{Decodificador/demultiplexor 1 de 8.} para demultiplexar ciertas %%@
l\'{\i}neas de direcci\'on del bus y as\'{\i} generar la se\~nal de habilitaci\'on de la %%@
tarjeta o dispositivo colocado en la ranura de expansi\'on. Esto tiene la desventaja de que la %%@
direcci\'on de la tarjeta se fija desde su dise\~no y podr\'{\i}a entrar en conflicto con otro %%@
dispositivo.

El m\'etodo de comparadores utiliza circuitos integrados de comparaci\'on, tales como el %%@
74LS85\footnote{Comparador de magnitud de 4 bits.} para comparar una direcci\'on fija (que %%@
puede ser modificada mediante interruptores\footnote{Del tipo {\it dip switch\/}.} o {\it %%@
jumpers\/} en la tarjeta) contra ciertas l\'{\i}neas del bus de direcciones. De esta manera, %%@
si la direcci\'on presente en el bus es igual a la fijada en la tarjeta, se genera la se\~nal %%@
de habilitaci\'on de la \'esta. Esto da la ventaja de que no se restringe la habilitaci\'on de %%@
la tarjeta a una direcci\'on permanente, sino que se puede modificar, en caso de que existiera %%@
conflicto con alg\'un otro dispositivo.

Los m\'etodos primero (l\'ogica aleatoria) y \'ultimo (memorias PROM de mapeo) no se %%@
consideraron en el dise\~no, ya que complican un tanto la l\'ogica de decodificaci\'on, al %%@
requerir muchos circuitos integrados. Sin embargo, cabe se\~nalar que la t\'ecnica de mapeo por %%@
memorias PROM tiene la ventaja de que un dispositivo puede ser accesado en cualquier %%@
direcci\'on, ya que dicha selecci\'on se lleva a cabo utilizando direcciones almacenadas en la %%@
PROM para mapear direcciones secuenciales dentro de localidades de memoria diferentes. Estas %%@
direcciones se pueden cambiar con solo cambiar la memoria PROM \cite{SADYR}.

%----------------------------------------------------------------------------

\subsection{Des\-crip\-ci\-\'on del cir\-cui\-to de de\-co\-di\-fi\-ca\-ci\-\'on\\ de %%@
SCIP.}
\label{Subsection:decodif}

Esta parte del circuito (la secci\'on de decodificaci\'on) se presenta en la hoja 2 de %%@
diagramas del Ap\'endice \ref{Apendice:diagramas}.

Se utiliza un dip switch de 8 bits (U5) para fijar la direcci\'on base de la tarjeta (300h), y %%@
mediante un comparador\footnote{La ventaja de usar el comparador es que la direcci\'on base %%@
de la tarjeta puede ser cambiada f\'acilmente.} 74LS684 (U2) se efectua la comparaci\'on entre %%@
la direcci\'on fijada por el dip switch y los bits presentes en las l\'{\i}neas A4 - A5 del %%@
bus de direcciones de la PC. Si ambos valores son iguales, la se\~nal %%@
$\overline{\mbox{PORTSEL}}$ es activada y usa como parte de la habilitaci\'on de un %%@
decodificador 74LS138 (U3). Si al mismo tiempo no est\'a presente la se\~nal AEN del bus de la %%@
PC\footnote{Esta se\~nal es activa en ALTO cuando hay acceso directo a memoria.}, el estado %%@
de las l\'{\i}neas $\overline{\mbox{IOR}}$ e $\overline{\mbox{IOW}}$ sirven para %%@
generar la se\~nal de selecci\'on de la tarjeta $\overline{\mbox{BOARDSEL}}$, la cual se %%@
activa si Y5 o Y6 de U3 est\'an en BAJO (es decir, si hay acceso a un puerto por parte de la CPU %%@
de la PC). Esto se logra mediante un AND en U4A (74LS08).

La se\-\~nal de $\overline{\mbox{BOARDSEL}}$ ha\-bi\-li\-ta en\-ton\-ces el de\-co\-%%@
di\-fi\-ca\-dor U1 (74LS138), el cu\-\'al sirve para generar, en conjunci\'on con las %%@
l\'{\i}neas de direcci\'on A0, A1 y A2 del bus de la PC, hasta 8 se\~nales para direccionar %%@
puertos en la tarjeta SCIP. Las salidas Y0 y Y1 ($\overline{\mbox{DATAREAD}}$ y %%@
$\overline{\mbox{WRITEDATA}}$ respectivamente, correspondientes a los puertos 300h y 301h %%@
de la computadora anfitriona) proporcionan las se\~nales para habilitar la lectura o escritura %%@
de datos. Dado que A3 se usa igualmente como habilitador de U1 (cuando A3=0), se puede colocar %%@
otro 74LS138 para tener hasta 16 puertos dentro de SCIP directamente (se podria usar l\'ogica %%@
adicional para direccionar m\'as puertos).

Para mantener la sincronizaci\'on entre SCIP y la PC se usa un circuito compuesto por los flip-%%@
flops 74LS74 U8A, U7A y U9A, los cuales, mediante las se\~nales de RESETDRV y CLOCK provenientes %%@
del bus de la PC, por cada selecci\'on de la tarjeta generan un pulso en la se\~nal %%@
$\overline{\mbox{IOCHRDY}}$ con duraci\'on de dos ciclos de reloj, lo cual genera un %%@
retardo en del bus de la PC anfitriona (Ver Secci\'on \ref{Section:ranuras}).

%----------------------------------------------------------------------------

\subsection{Des\-crip\-ci\-\'on del cir\-cui\-to de comunicaci\'on\\de SCIP.}
\label{Subsection:comunicacion}

La parte del circuito de SCIP que realiza la comunicaci\'on entre la PC y la tarjeta se muestra %%@
en la hoja de diagramas n\'umero 3 del  Ap\'endice \ref{Apendice:diagramas}.

El circuito integrado que es el encargado principal de la comunicaci\'on es el controlador %%@
programable de perif\'ericos 8255 (ver Ap\'endice \ref{Apendice:chips} y Figura %%@
\ref{Figura:8255}). Este es un dispositivo muy vers\'atil ya que puede ser programado en tres %%@
modos principales de operaci\'on (Modos 0, 1 y 2), los cuales a su vez pueden ser combinados. %%@
Cualquiera de estos modos pueden ser seleccionados mediante software en cualquier momento. %%@
Cuenta con 3 puertos de 8 bits cada uno (A, B y C), los cuales se seleccionan mediante las %%@
l\'{\i}neas A0 y A1 del bus interno de direcciones. El Puerto A junto con la parte superior del %%@
Puerto C (C7 - C4) conforman el Grupo A, mientras que el Grupo B esta compuesto por el Puerto B %%@
y la parte inferior del Puerto C (C3 - C0)

\vspace{12pt}

\begin{figure}[!htb]
\vskip 5mm
\special{center
         8255.gif,                
         \the\hsize 78mm}
\vskip 80mm
\caption{Controlador Programable de Perif\'ericos 8255.} 
\label{Figura:8255}
\end{figure}

\vspace{12pt}

La parte del dise\~no a continuaci\'on descrita se puede apreciar en la hoja de diagramas %%@
n\'umero 5. En dicho diagrama, U6 (un 74LS138) mediante un AND de las l\'{\i}neas Y4, Y5, Y6 y %%@
Y7 genera la se\~nal $\overline{\mbox{PPISEL}}$  la cual se conecta a la terminal %%@
$\overline{\mbox{CS}}$ del 8255 para habilitarlo. Las se\~nales generadas por el 8288 para %%@
el control de entrada y salida ($\overline{\mbox{IOR}}$ y $\overline{\mbox{IOW}}$) son %%@
utilizadas para controlar la entrada y salida del 8255. Las l\'{\i}neas del bus  de datos %%@
interno de SCIP est\'an conectadas a las terminales 27 a 34 (D7 - D0) del 8255, mientras que las %%@
terminales correspondientes al Puerto A del PPI estan conectadas a un transceiver 74LS245 (U4) %%@
el cual es usado para comunicarse con el bus del sistema anfitri\'on. La direcci\'on de este %%@
transceiver es controlada con la l\'{\i}nea $\overline{\mbox{DATAREAD}}$ generada por la %%@
l\'ogica de decodificaci\'on presentada en la hoja 2 de diagramas, y que corresponde al puerto %%@
de lectura (puerto 300h de la PC), mientras que la habilitaci\'on de dicho transceiver se lleva %%@
a cabo con la l\'{\i}nea $\overline{\mbox{BOARDSEL}}$ tambi\'en generada internamente %%@
(vease Secci\'on \ref{Section:decodifbus}). Esto posibilita la lectura de datos de 8 bits %%@
desde la computadora anfitriona.

Para el caso de escritura de un byte a SCIP, se conect\'o al puerto B del 8255 (terminales 18 a %%@
25) un 74LS373\footnote{Latch transparente de 8 bits con salidas de 3 estados.} cuyas entradas %%@
estan conectadas al bus de datos de la computadora anfitriona. La terminal de habilitaci\'on de %%@
este latch esta conectada a la l\'{\i}nea $\overline{\mbox{WRITEDATA}}$ correspondiente al %%@
puerto 301h de la PC (ver Subsecci\'on \ref{Subsection:decodif}) y la terminal de %%@
habilitaci\'on de salidas esta a tierra. Esto posibilita la escritura de un byte a SCIP por %%@
parte de la PC anfitriona.

Se usa la se\~nal de $\overline{\mbox{MASTERESET}}$ generada por el 8284 para reinicializar %%@
al 8255.  Cuando el 8255 recibe dicha se\~nal (es decir, cuando la terminal 35 es puesta en %%@
ALTO), todos los puertos son colocados en el modo de entrada (todas las l\'{\i}neas estaran en %%@
estado de alta impedancia) \cite{Intel:Perif}. Este dispositivo puede permanecer en tal estado %%@
si no se programa.

\begin{figure}[!htb]\centering
%TexCad Options 
%\grade{\off} 
%\emlines{\off} 
%\beziermacro{\on} 
%\reduce{\on} 
%\snapping{\off} 
%\quality{2.00} 
%\graddiff{0.01} 
%\snapasp{1} 
%\zoom{1.00} 
\unitlength 1mm 
\linethickness{0.4pt} 
\begin{picture}(48.33,65.00) 
\put(14.33,5.00){\framebox(18.00,39.33)[cc]{8255A }} 
\put(30.00,40.33){\makebox(0,0)[cc]{A}} 
\put(30.00,7.67){\makebox(0,0)[cc]{B}} 
\put(27.00,20.33){\makebox(0,0)[cc]{C}} 
\put(29.00,25.00){\line(1,2){2.17}} 
\put(29.00,25.00){\line(1,-2){2.17}} 
\put(8.33,25.00){\vector(1,0){6.00}} 
\put(10.00,25.00){\vector(-1,0){3.67}} 
\put(4.00,27.67){\makebox(0,0)[lc]{D0-D7}} 
\put(32.33,40.67){\vector(1,0){13.33}} 
\put(46.00,8.00){\vector(-1,0){13.67}} 
\put(45.67,29.33){\vector(-1,0){13.00}} 
\put(32.33,20.67){\vector(1,0){13.33}} 
\put(37.33,39.00){\line(4,3){4.00}} 
\put(37.67,27.67){\line(6,5){3.00}} 
\put(37.67,18.67){\line(6,5){4.00}} 
\put(38.00,6.33){\line(4,3){4.33}} 
\put(42.67,42.33){\makebox(0,0)[cc]{8}} 
\put(42.33,31.00){\makebox(0,0)[cc]{4}}  
\put(42.33,22.33){\makebox(0,0)[cc]{4}} 
\put(43.00,9.67){\makebox(0,0)[cc]{8}} 
\put(48.33,40.67){\makebox(0,0)[lc]{PA7-PA0}} 
\put(48.00,29.33){\makebox(0,0)[lc]{PC7-PC4}} 
\put(48.00,20.67){\makebox(0,0)[lc]{PC3-PC0}} 
\put(48.00,8.00){\makebox(0,0)[lc]{PB7-PB0}} 
\put(5.00,60.00){\framebox(5.00,5.00)[cc]{1}} 
\put(10.00,60.00){\framebox(5.00,5.00)[cc]{0} } 
\put(15.00,60.00){\framebox(5.00,5.00)[cc]{0}} 
\put(20.00,59.67){\framebox(5.00,5.33)[cc]{0}} 
\put(25.00,60.00){\framebox(5.00,5.00)[cc]{1}} 
\put(30.00,60.00){\framebox(5.00,5.00)[cc]{0}} 
\put(35.00,60.00){\framebox(5.00,5.00)[cc]{1}} 
\put(40.00,60.00){\framebox(5.00,5.00)[cc]{0}} 
\put(7.00,55.00){\makebox(0,0)[cc]{D7}} 
\put(12.67,55.00){\makebox(0,0)[cc]{D6}} 
\put(17.33,55.00){\makebox(0,0)[cc]{D5}}  
\put(22.33,55.00){\makebox(0,0)[cc]{D4}} 
\put(27.33,54.67){\makebox(0,0)[cc]{D3}} 
\put(32.33,55.00){\makebox(0,0)[cc]{D2}} 
\put(37.67,55.00){\makebox(0,0)[cc]{D1}} 
\put(42.33,55.00){\makebox(0,0)[cc]{D0}} 
\end{picture}
\caption{Controlador Programable de Perif\'ericos 8255 en configuraci\'on de Modo 0, con la %%@
palabra de palabra de control 6, programando el Puerto A como salida y el Puerto B como %%@
entrada.}
\label{Figure:8255modo0}
\end{figure}

El 8255, para los prop\'ositos de este trabajo, se program\'o en Modo 0. Este modo brinda %%@
operaciones de entrada y salida simples, para cada uno de los tres puertos del 8255. No existen %%@
se\~nales de {\it ``handshaking''\/}, sino que los datos son simplemente leidos o escritos %%@
mediante puertos. Este modo consiste en lo siguiente:

\begin{itemize}
\item Dos puertos de 8 bits y dos de 4 bits.
\item Cualquier puerto puede ser entrada o salida.
\item Las entradas no tienen latches.
\item Las salidas tienen latches.
\item Existen 16 posibles combinaciones de entrada/salida en este modo.
\end{itemize}

En la Tabla \ref{Tabla:program8255} se muestra la operaci\'on b\'asica del 8255 %%@
\cite{Intel:Perif}.

\vspace{12pt} 
 
% Table created by WinTeX 95: 6 Columns x 13 Rows. 
\begin{table}[!htb]\centering 
\begin{tabular}{|l|l|l|l|l|l|} 
%Row: 1 
\cline{1-6} 
\vbox to1.50ex{\vspace{1pt}\vfil\hbox to4.20ex{\hfil A1\hfil}\vfil} &  
\vbox to1.50ex{\vspace{1pt}\vfil\hbox to4.20ex{\hfil A0\hfil}\vfil} &  
\vbox to1.50ex{\vspace{1pt}\vfil\hbox to4.60ex{\hfil RD\hfil}\vfil} &  
\vbox to1.50ex{\vspace{1pt}\vfil\hbox to5.00ex{\hfil WR\hfil}\vfil} &  
\vbox to1.50ex{\vspace{1pt}\vfil\hbox to4.00ex{\hfil CS\hfil}\vfil} &  
\vbox to1.50ex{\vspace{1pt}\vfil\hbox to31.20ex{\hfil Op. de Entrada %%@
(Lectura)\hfil}\vfil} \\ 
 
%Row: 2 
\cline{1-6} 
\vbox to1.70ex{\vspace{1pt}\vfil\hbox to4.20ex{\hfil 0\hfil}\vfil} &  
\vbox to1.70ex{\vspace{1pt}\vfil\hbox to4.20ex{\hfil 0\hfil}\vfil} &  
\vbox to1.70ex{\vspace{1pt}\vfil\hbox to4.60ex{\hfil 0\hfil}\vfil} &  
\vbox to1.70ex{\vspace{1pt}\vfil\hbox to5.00ex{\hfil 1\hfil}\vfil} &  
\vbox to1.70ex{\vspace{1pt}\vfil\hbox to4.00ex{\hfil 0\hfil}\vfil} &  
\vbox to1.70ex{\vspace{1pt}\vfil\hbox to31.20ex{\hfil Puerto A $\Longrightarrow$ Bus %%@
de Datos\hfil}\vfil} \\ 
 
%Row: 3 
\cline{1-6} 
\vbox to1.70ex{\vspace{1pt}\vfil\hbox to4.20ex{\hfil 0\hfil}\vfil} &  
\vbox to1.70ex{\vspace{1pt}\vfil\hbox to4.20ex{\hfil 1\hfil}\vfil} &  
\vbox to1.70ex{\vspace{1pt}\vfil\hbox to4.60ex{\hfil 0\hfil}\vfil} &  
\vbox to1.70ex{\vspace{1pt}\vfil\hbox to5.00ex{\hfil 1\hfil}\vfil} &  
\vbox to1.70ex{\vspace{1pt}\vfil\hbox to4.00ex{\hfil 0\hfil}\vfil} &  
\vbox to1.70ex{\vspace{1pt}\vfil\hbox to31.20ex{\hfil Puerto B $\Longrightarrow$ Bus %%@
de Datos\hfil}\vfil} \\ 
 
%Row: 
\cline{1-6}
\vbox to1.70ex{\vspace{1pt}\vfil\hbox to4.20ex{\hfil 1\hfil}\vfil} & 
\vbox to1.70ex{\vspace{1pt}\vfil\hbox to4.20ex{\hfil 0\hfil}\vfil} & 
\vbox to1.70ex{\vspace{1pt}\vfil\hbox to4.60ex{\hfil 0\hfil}\vfil} & 
\vbox to1.70ex{\vspace{1pt}\vfil\hbox to5.00ex{\hfil 1\hfil}\vfil} & 
\vbox to1.70ex{\vspace{1pt}\vfil\hbox to4.00ex{\hfil 0\hfil}\vfil} & 
\vbox to1.70ex{\vspace{1pt}\vfil\hbox to31.20ex{\hfil Puerto C $\Longrightarrow$ Bus %%@
de Datos\hfil}\vfil} \\

%Row: 5
\cline{1-6}
\vbox to1.70ex{\vspace{1pt}\vfil\hbox to4.20ex{\hfil \hfil}\vfil} & 
\vbox to1.70ex{\vspace{1pt}\vfil\hbox to4.20ex{\hfil \hfil}\vfil} & 
\vbox to1.70ex{\vspace{1pt}\vfil\hbox to4.60ex{\hfil \hfil}\vfil} & 
\vbox to1.70ex{\vspace{1pt}\vfil\hbox to5.00ex{\hfil \hfil}\vfil} & 
\vbox to1.70ex{\vspace{1pt}\vfil\hbox to4.00ex{\hfil \hfil}\vfil} & 
\vbox to1.70ex{\vspace{1pt}\vfil\hbox to31.20ex{\hfil Op. de Salida %%@
(Escritura)\hfil}\vfil} \\

%Row: 6
\cline{1-6}
\vbox to1.70ex{\vspace{1pt}\vfil\hbox to4.20ex{\hfil 0\hfil}\vfil} & 
\vbox to1.70ex{\vspace{1pt}\vfil\hbox to4.20ex{\hfil 0\hfil}\vfil} & 
\vbox to1.70ex{\vspace{1pt}\vfil\hbox to4.60ex{\hfil 1\hfil}\vfil} & 
\vbox to1.70ex{\vspace{1pt}\vfil\hbox to5.00ex{\hfil 0\hfil}\vfil} & 
\vbox to1.70ex{\vspace{1pt}\vfil\hbox to4.00ex{\hfil 0\hfil}\vfil} & 
\vbox to1.70ex{\vspace{1pt}\vfil\hbox to31.20ex{\hfil Bus de Datos $\Longrightarrow$ %%@
Puerto A\hfil}\vfil} \\

%Row: 7
\cline{1-6}
\vbox to1.70ex{\vspace{1pt}\vfil\hbox to4.20ex{\hfil 0\hfil}\vfil} & 
\vbox to1.70ex{\vspace{1pt}\vfil\hbox to4.20ex{\hfil 1\hfil}\vfil} & 
\vbox to1.70ex{\vspace{1pt}\vfil\hbox to4.60ex{\hfil 1\hfil}\vfil} & 
\vbox to1.70ex{\vspace{1pt}\vfil\hbox to5.00ex{\hfil 0\hfil}\vfil} & 
\vbox to1.70ex{\vspace{1pt}\vfil\hbox to4.00ex{\hfil 0\hfil}\vfil} & 
\vbox to1.70ex{\vspace{1pt}\vfil\hbox to31.20ex{\hfil Bus de Datos $\Longrightarrow$ %%@
Puerto B\hfil}\vfil} \\

%Row: 8
\cline{1-6}
\vbox to1.70ex{\vspace{1pt}\vfil\hbox to4.20ex{\hfil 1\hfil}\vfil} & 
\vbox to1.70ex{\vspace{1pt}\vfil\hbox to4.20ex{\hfil 0\hfil}\vfil} & 
\vbox to1.70ex{\vspace{1pt}\vfil\hbox to4.60ex{\hfil 1\hfil}\vfil} & 
\vbox to1.70ex{\vspace{1pt}\vfil\hbox to5.00ex{\hfil 0\hfil}\vfil} & 
\vbox to1.70ex{\vspace{1pt}\vfil\hbox to4.00ex{\hfil 0\hfil}\vfil} & 
\vbox to1.70ex{\vspace{1pt}\vfil\hbox to31.20ex{\hfil Bus de Datos $\Longrightarrow$ %%@
Puerto C\hfil}\vfil} \\

%Row: 9
\cline{1-6}
\vbox to1.70ex{\vspace{1pt}\vfil\hbox to4.20ex{\hfil 1\hfil}\vfil} & 
\vbox to1.70ex{\vspace{1pt}\vfil\hbox to4.20ex{\hfil 1\hfil}\vfil} & 
\vbox to1.70ex{\vspace{1pt}\vfil\hbox to4.60ex{\hfil 1\hfil}\vfil} & 
\vbox to1.70ex{\vspace{1pt}\vfil\hbox to5.00ex{\hfil 0\hfil}\vfil} & 
\vbox to1.70ex{\vspace{1pt}\vfil\hbox to4.00ex{\hfil 0\hfil}\vfil} & 
\vbox to1.70ex{\vspace{1pt}\vfil\hbox to31.20ex{\hfil Bus de Datos $\Longrightarrow$ %%@
Control\hfil}\vfil} \\

%Row: 10
\cline{1-6}
\vbox to1.70ex{\vspace{1pt}\vfil\hbox to4.20ex{\hfil \hfil}\vfil} & 
\vbox to1.70ex{\vspace{1pt}\vfil\hbox to4.20ex{\hfil \hfil}\vfil} & 
\vbox to1.70ex{\vspace{1pt}\vfil\hbox to4.60ex{\hfil \hfil}\vfil} & 
\vbox to1.70ex{\vspace{1pt}\vfil\hbox to5.00ex{\hfil \hfil}\vfil} & 
\vbox to1.70ex{\vspace{1pt}\vfil\hbox to4.00ex{\hfil \hfil}\vfil} & 
\vbox to1.70ex{\vspace{1pt}\vfil\hbox to31.20ex{\hfil Deshabilitaci\'on\hfil}\vfil} %%@
\\

%Row: 11
\cline{1-6}
\vbox to1.70ex{\vspace{1pt}\vfil\hbox to4.20ex{\hfil X\hfil}\vfil} & 
\vbox to1.70ex{\vspace{1pt}\vfil\hbox to4.20ex{\hfil X\hfil}\vfil} & 
\vbox to1.70ex{\vspace{1pt}\vfil\hbox to4.60ex{\hfil X\hfil}\vfil} & 
\vbox to1.70ex{\vspace{1pt}\vfil\hbox to5.00ex{\hfil X\hfil}\vfil} & 
\vbox to1.70ex{\vspace{1pt}\vfil\hbox to4.00ex{\hfil 1\hfil}\vfil} & 
\vbox to1.70ex{\vspace{1pt}\vfil\hbox to31.20ex{\hfil Bus de Datos $\Longrightarrow$ %%@
3er. Estado\hfil}\vfil} \\

%Row: 12
\cline{1-6}
\vbox to1.70ex{\vspace{1pt}\vfil\hbox to4.20ex{\hfil 1\hfil}\vfil} & 
\vbox to1.70ex{\vspace{1pt}\vfil\hbox to4.20ex{\hfil 1\hfil}\vfil} & 
\vbox to1.70ex{\vspace{1pt}\vfil\hbox to4.60ex{\hfil 0\hfil}\vfil} & 
\vbox to1.70ex{\vspace{1pt}\vfil\hbox to5.00ex{\hfil 1\hfil}\vfil} & 
\vbox to1.70ex{\vspace{1pt}\vfil\hbox to4.00ex{\hfil 0\hfil}\vfil} & 
\vbox to1.70ex{\vspace{1pt}\vfil\hbox to31.20ex{\hfil Condici\'on Ilegal\hfil}\vfil} %%@
\\

%Row: 13
\cline{1-6}
\vbox to1.70ex{\vspace{1pt}\vfil\hbox to4.20ex{\hfil X\hfil}\vfil} & 
\vbox to1.70ex{\vspace{1pt}\vfil\hbox to4.20ex{\hfil X\hfil}\vfil} & 
\vbox to1.70ex{\vspace{1pt}\vfil\hbox to4.60ex{\hfil 1\hfil}\vfil} & 
\vbox to1.70ex{\vspace{1pt}\vfil\hbox to5.00ex{\hfil 1\hfil}\vfil} & 
\vbox to1.70ex{\vspace{1pt}\vfil\hbox to4.00ex{\hfil 0\hfil}\vfil} & 
\vbox to1.70ex{\vspace{1pt}\vfil\hbox to31.20ex{\hfil Bus de Datos $\Longrightarrow$ %%@
3er. Estado\hfil}\vfil} \\

\cline{1-6}
\end{tabular}
\caption{Modo B\'asico de Operaci\'on del 8255.}
\label{Tabla:program8255}
\end{table}

Cuando se escribe un dato al 8255 a trav\'es del bus de datos, y $A1 = A0 = 1$, entonces se %%@
transfiere una palabra de control al 8255, program\'andolo (ver Tabla %%@
\ref{Tabla:program8255}).

En las pruebas realizadas se program\'o al 8255 utilizando la palabra de control 6 (ver %%@
Ap\'endice \ref{Apendice:chips}), de tal forma que el Puerto A trabaja como salida y el Puerto %%@
B como entrada, ademas de que el Puerto C se divide en 2 partes: PC7-PC4 funcionando como %%@
entrada y PC3-PC0 como salida. El c\'odigo que realiza esta programaci\'on es el siguiente:

\begin{verbatim}
   MOV DX,3C7h    ; Direccion del pto. de control del 8255    
                  ; (A0=A1=1)
   MOV AL,8Ah     ; Modo 0, Pto. A = Salida, Pto. B = Entrada, 
                  ; PC03 = Salida, PC47 = Entrada
   OUT DX,AL      ; Programar el 8255
\end{verbatim}

De esta forma el 8255 sirve para comunicar a la tarjeta con la computadora, viendo a esta como a %%@
un perif\'erico de SCIP.

%----------------------------------------------------------------------------

\section{Protocolo de comunicaci\'on entre la \\tarjeta SCIP y la computadora anfitriona}
\label{Seccion:protocolocomunicacion}

Dado que la tarjeta SCIP es una tarjeta coprocesadora de prop\'osito general, a cuyo dise\~no %%@
b\'asico se le pueden agregar otros componenetes, el protocolo de comandos de comunicaci\'on %%@
entre la tarjeta y la PC puede variar dependiendo de la funci\'on de SCIP. Un protocolo %%@
b\'asico propuesto para llevar a cabo la comunicaci\'on se presenta a continuaci\'on.

\begin{table}[!htb]\centering
\begin{tabular}{|c|c|l|}\hline
Dec. & Binario & Prop\'osito \\ \hline
1 & 00000001 & \parbox{8cm}{\vspace{3pt}Iniciar programa de usuario ubicado en la EPROM de %%@
SCIP.\vspace{3pt}} \\ \hline
2 & 00000010 & \parbox{8cm}{\vspace{3pt}Iniciar transferencia de datos de la PC a SCIP (los %%@
siguientes 2 bytes en el Puerto B indican cuantos bytes ser\'an transferidos).\vspace{3pt}} %%@
\\ \hline
3 & 00000011 & \parbox{8cm}{\vspace{3pt}En espera de transferencia de datos de SCIP a la PC %%@
anfitriona.\vspace{3pt}} \\ \hline
4 & 00000100 & \parbox{8cm}{\vspace{3pt}PC anfitriona lista para transferencia de datos de %%@
la tarjeta a la PC (los primeros 2 bytes en el Puerto A le indican a la PC cuantos bytes se %%@
transferiran).\vspace{3pt}} \\ \hline
\end{tabular}
\caption{Comandos b\'asicos para el SCIP.}
\label{Tabla:SCIPcomandos}
\end{table}

Dado que el Puerto B del PPI 8255 se programa en Modo 0 (ver Subsecci\'on %%@
\ref{Subsection:comunicacion}), este es usado por la computadora anfitriona para escribir %%@
comandos a la tarjeta, mientras que el Puerto A es utilizado para transferencia de comandos y %%@
datos hacia aquella. Los comandos b\'asicos de comunicaci\'on para SCIP se presentan en la %%@
Tabla \ref{Tabla:SCIPcomandos}.

Este conjunto de comandos b\'asicos se puede ampliar dependiendo de la aplicaci\'on y de los %%@
programas que la tarjeta ejecute.

A su vez, la tarjeta SCIP puede indicar su estado ya sea a trav\'es del Puerto A o de PC0-PC2, o %%@
bien mediante el uso de las se\~nales de {\it handshaking\/} del 8255, en caso de que se %%@
programe en Modo 1 o Modo 2.

%----------------------------------------------------------------------------

\section{Consideraciones importantes sobre la programaci\'on.}
\label{Section:consideraciones}

Para la realizaci\'on de los programas que se colocan en SCIP es preciso se\~nalar los %%@
siguientes aspectos, poniendo como ejemplo el siguiente fragmento que inicializa un programa en %%@
SCIP:

\begin{listing}{1}
   ORG 0FFFF0h

 START:
   MOV AX,0F000h
   MOV CS,AX
   JMP PROG
\end{listing} 

\begin{itemize}

\item Localizaci\'on del programa en memoria: El programa inicial de SCIP siempre debe comenzar %%@
en la direcci\'on FFFF0h absoluta, es decir, en el segmento y desplazamiento F000:FFF0 hex., ya %%@
que el procesador salta a esa direcci\'on al inicializarse. Esto se realiza con la pseu\-do\-%%@
ins\-truc\-ci\'on {\bf ORG} de ensamblador, como se ve en la l\'{\i}nea 1 del c\'odigo %%@
arriba mostrado.

\item Inicializacion del segmento de c\'odigo: En las l\'{\i}neas 4 y 5 del c\'odigo arriba %%@
presentado se puede ver la manera en como inicializar el segmento de c\'odigo del programa de %%@
inicio de SCIP. Se tiene que poner un F000h en el registro CS ya que de otra forma el programa %%@
se ``pierde'' al realizar el salto al cuerpo del programa de inicializaci\'on (l\'{\i}nea 6 %%@
del listado).
\end{itemize}

As\'{\i} mismo el cuerpo del programa de inicializaci\'on debe ir de preferencia en la %%@
direcci\'on 100h del segmento F000, realizando esto como se muestra en la l\'{\i}nea 4 del %%@
siguiente listado parcial:

\begin{listing}{1}
CODIGO SEGMENT 'Code'
   ASSUME CS:CODIGO, DS:CODIGO, ES:NOTHING, SS:NOTHING

   ORG 0F0100h

PROG PROC NEAR  
; el cuerpo principal se coloca aqui...
\end{listing}

Esto es porque generalmente el ensamblador rechaza direcciones de inicio de un programa ubicadas %%@
en un desplazamiento menor que 100h.

%----------------------------------------------------------------------------

