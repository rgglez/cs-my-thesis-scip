\chapter[Manual del Usuario.]{Manual del Usuario.}
\label{Apendice:manual}

%----------------------------------------------------------------------------

\fancyhead[RE]{\slshape SCIP - Ap. \thechapter~ - Manual del Usuario.}
\fancyhead[LO]{\slshape Rodolfo Gonz\'alez Gonz\'alez.}
\fancyhead[LE,RO]{\thepage}

%----------------------------------------------------------------------------

\section{Introducci\'on.}
\label{Section:introduction}

En este manual del usuario de SCIP se presentan los procedimientos para:

\begin{enumerate}
\item Ma\-ne\-jo e ins\-ta\-la\-ci\-\'on del hard\-wa\-re de SCIP.
\item Realizaci\'on de software para SCIP.
\end{enumerate}

%----------------------------------------------------------------------------

\section{Ma\-ne\-jo e ins\-ta\-la\-ci\-\'on del hard\-wa\-re de SCIP.}
\label{Section:manualhdw}

En esta secci\'on se detallan los procedimientos para la instalaci\'on y manejo de la tarjeta %%@
SCIP.

%----------------------------------------------------------------------------

\subsection{Introducci\'on al hadware de SCIP.}
\label{Subsection:introhwd}

SCIP es un sistema de hadware que consiste de una tarjeta de expansi\'on para bus ISA AT que %%@
puede ser utilizada en computadoras basadas en el Intel 286 o superior. El prototipo de esta %%@
tarjeta esta ensamblado en una placa para wire-wrap de 16 bits JDR-PR10 de JDR Microdevices.

\begin{figure}[!htb]
\vskip 5mm
\special{center
         foto1.gif,                
         \the\hsize 78mm}
\vskip 78mm
\caption{Lado de componentes de SCIP.} 
\label{Figura:fotoSCIP1}
\end{figure}

\begin{figure}[!hbt]
\vskip 5mm
\special{center
         foto2.gif,                
         \the\hsize 78mm}
\vskip 78mm
\caption{Lado de alambramiento del prototipo de SCIP.} 
\label{Figura:fotoSCIP2}
\end{figure}

En  la Figura \ref{Figura:fotoSCIP1} se puede ver una vista del lado de componentes de SCIP, %%@
mientras que en la Figura \ref{Figura:fotoSCIP2} se aprecia el lado de alambramiento del %%@
sistema.

\begin{figure}[!htb]\centering
%TexCad Options
%\grade{\off}
%\emlines{\off}
%\beziermacro{\on}
%\reduce{\on}
%\snapping{\off}
%\quality{2.00}
%\graddiff{0.01}
%\snapasp{1}
%\zoom{1.00}
\unitlength 1mm
\linethickness{0.4pt}
\begin{picture}(150.33,70.00)
\put(4.67,10.00){\framebox(145.67,60.00)[cc]{ }}
\put(96.33,3.00){\framebox(52.67,7.00)[cc]{ }}
\put(71.67,3.00){\framebox(21.00,7.00)[cc]{ }}
\put(70.67,12.00){\framebox(73.33,25.33)[cc]{1}}
\put(7.33,42.00){\framebox(40.00,25.67)[cc]{2}}
\put(50.00,42.00){\framebox(22.33,25.67)[cc]{3}}
\put(7.33,12.00){\framebox(12.00,26.67)[cc]{4}}
\put(22.00,12.00){\framebox(25.33,26.67)[cc]{5}}
\put(50.00,12.00){\framebox(18.67,26.33)[cc]{6}}
\put(75.67,42.00){\framebox(12.33,25.67)[cc]{ROM1}}
\put(90.67,42.00){\framebox(13.00,25.67)[cc]{ROM2}}
\put(106.33,42.00){\framebox(37.67,25.67)[cc]{7}}
\end{picture}
\caption{Diagrama en bloques de SCIP: (1) Secci\'on preconstruida de decodificaci\'on, (2) %%@
M\'odulo de Procesamiento y Control de BUS, (3) M\'o\-du\-lo de de\-co\-di\-fi\-ca\-ci\-%%@
\'on in\-ter\-na y bu\-ffers, (4) M\'odulo de temporizaci\'on, (5) M\'odulo de %%@
comunicaci\'on, (6) M\'odulo de l\'ogica de selecci\'on, (7) M\'odulo de memorias RAM, (ROM1) %%@
Memoria EPROM 1, (ROM2) Memoria EPROM 2.} 
\label{Figura:partes}
\end{figure}

En la Figura \ref{Figura:partes} se muestra un esquema de las partes constituyentes de SCIP. La %%@
descripci\'on de estas partes es la siguiente\footnote{Los n\'umeros se refieren a los que %%@
identifican cada bloque en la figura.}:

\begin{description}
\item[1] - {\bf Secci\'on preconstruida de decodificaci\'on}. De esta secci\'on solo se %%@
utilizaron los transceivers para el bus de direcciones.
\item[2] - {\bf M\'odulo de Procesamiento y Control de BUS}. Esta secci\'on contiene al %%@
procesador 8088 y el coprocesador 8087, as\'{\i} como el controlador del bus 8288, y se encarga %%@
del procesamiento.
\item[3] - {\bf M\'odulo de decodificaci\'on interna y buffers}. Esta secci\'on contiene la %%@
l\'ogica de decodificaci\'on de puertos y memoria interna as\'{\i} como buffers y transceivers %%@
para las memorias.
\item[4] - {\bf M\'odulo de temporizaci\'on}. Esta secci\'on contiene la l\'ogica de %%@
temporizaci\'on y reinicializaci\'on de la tarjeta.
\item[5] - {\bf M\'odulo de comunicaci\'on}. Esta secci\'on contiene al Controlador %%@
Programable de Perif\'ericos 8255 as\'{\i} como buffers y transceivers para la comunicaci\'on %%@
entre el bus de expansi\'on de la computadora anfitriona y el bus de SCIP.
\item[6] - {\bf M\'odulo de l\'ogica de selecci\'on}. Esta secci\'on se encarga de %%@
decodificar las direcciones y se\~nales del bus de la computadora anfitriona para realizar la %%@
selecci\'on de la tarjeta SCIP y los diversos puertos que la conforman.
\item[7] - {\bf M\'odulo de memorias RAM}. Esta secci\'on contiene a las cuatro memorias RAM %%@
est\'aticas de 32Kb.
\item[ROM1] - {\bf Memoria EPROM 1}. En esta memoria esta la direcci\'on de arranque del %%@
procesador 8088 (F000:FFF0 hex), por lo que en esta memoria se debe colocar el programa de %%@
inicio de la tarjeta, en la direcci\'on adecuada, como se describe en la Secci\'on %%@
\ref{Section:prograSCIP}.
\item[ROM2] - {\bf Memoria EPROM 2}. Esta es la segunda memoria EPROM, y puede contener el %%@
programa de usuario, en caso de que la ROM1 solo contenga la rutina de inicializaci\'on.
\end{description}

%----------------------------------------------------------------------------

\subsection{Precauciones.}
\label{Subsection:precauciones}

Dado que se utilizan circuitos MOS y CMOS en la tarjeta SCIP, los cuales son altamente sensibles %%@
a descargas de electricidad est\'atica, se recomienda manejar la tarjeta con precauci\'on.

\begin{itemize}
\item La computadora debe estar apagada al conectar o desconectar la tarjeta de expansi\'on.
\item No retire la tarjeta de su empaque antiest\'atico hasta que se vaya a utilizar.
\item Procure evitar tocar las terminales de las bases e wire-wrap de la tarjeta prototipo, %%@
as\'{\i} como los componentes.
\item Descarguese de electricidad est\'atica tocando una parte no pintada del chasis de la %%@
computadora donde vaya a instalar la tarjeta, o cualquier otra pieza met\'alica que este %%@
aterrizada.
\end{itemize}

%----------------------------------------------------------------------------

\subsection{Instalaci\'on.}
\label{Subsection:instalacion}

Para instalar la tarjeta SCIP en la computadora anfitriona, siga estos pasos:

\begin{enumerate}
\item Desconecte la computadora de la corriente.
\item Abra el gabinete de la computadora.
\item Busque una ranura de expansi\'on ISA libre.
\item Inserte firmemente la tarjeta SCIP en dicha ranura.
\end{enumerate}

Una vez hecho esto podr\'a encender la computadora. 

\begin{figure}[!hbt]
\vskip 5mm
\special{center
         montaje.gif,                
         \the\hsize 78mm}
\vskip 78mm
\caption{Tarjeta de expansi\'on SCIP montada sobre una tarjeta extensora del bus (no necesaria %%@
en la pr\'actica) en un ensamble 386, durante las pruebas del sistema.} 
\label{Figura:montaje}
\end{figure}

En la Figura \ref{Figura:montaje} se presenta una fotograf\'{\i}a de la tarjeta de %%@
expansi\'on montada en una computadora. En ese caso se utiliz\'o una tarjeta extensora del bus, %%@
para que se pudiera apreciar mejor, pero en la pr\'actica no es necesario.

%----------------------------------------------------------------------------

\subsection{Manejo del Hadware.}
\label{Subsection:manejo}

La tarjeta SCIP solo tiene un control operable directamente. Este es el bot\'on de {\it %%@
RESET\/} ubicado en la parte inferior izquierda de la tarjeta. Este funciona con un ``push %%@
button'', el cual al presionarse hace que el procesador de SCIP reinicie y se vuelva a comenzar %%@
el programa ubicado en la memoria EPROM.

%----------------------------------------------------------------------------

\section{Programaci\'on de SCIP.}
\label{Section:prograSCIP}

SCIP tiene dos memorias EPROM de 64Kb y 4 memorias RAM de 32Kb. El programa de inicio de SCIP %%@
debe colocarse en la EPROM 1 (ver Subsecci\'on \ref{Subsection:introhwd}), ya que en esta %%@
memoria se mapea la direcci\'on de arranque del procesador 8088, la cual es F000:FFF0 %%@
hexadecimal.

En el Cap\'{\i}tulo \ref{Capitulo:diseno} se presenta la descripci\'on completa de los %%@
bloques de memoria y puertos utilizados por el sistema.

Los programas de SCIP se pueden hacer en cualquier lenguaje que genere c\'odigo para el 8088, %%@
pero se recomienda el uso de ensamblador, ya que permite la generaci\'on de c\'odigo compacto y %%@
es m\'as versatil para el manejo de los registros de segmento y del hardware en general.

%----------------------------------------------------------------------------

\subsection{Localizaci\'on del programa en memoria.}
\label{Section:locprog}

Para la realizaci\'on de los programas que se colocan en SCIP es preciso se\~nalar los %%@
siguientes aspectos, poniendo como ejemplo el siguiente fragmento que inicializa un programa en %%@
SCIP:

\begin{listing}{1}
   ORG 0FFFF0h

 START:
   MOV AX,0F000h
   MOV CS,AX
   JMP PROG
\end{listing} 

\begin{itemize}
\item Localizaci\'on del programa en memoria: El programa inicial de SCIP siempre debe comenzar %%@
en la direcci\'on FFFF0h absoluta, es decir, en el segmento F000h y el desplazamiento FFF0h, ya %%@
que el procesador salta a esa direcci\'on al inicializarse. Esto se realiza con la pseu\-%%@
doins\-truc\-ci\'on {\bf ORG} de ensamblador, como se ve en la l\'{\i}nea 1 del c\'odigo %%@
arriba mostrado.

\item Inicializacion del segmento de c\'odigo: En las l\'{\i}neas 4 y 5 del c\'odigo arriba %%@
presentado se puede ver la manera en como inicializar el segmento de c\'odigo del programa de %%@
inicio de SCIP. Se tiene que poner un F000h en el registro CS ya que de otra forma el programa %%@
se ``pierde''\footnote{Se present\'o \'este problema durante las pruebas del sistema, pero se %%@
resolvi\'o con lo que se explica aqui.} al realizar el salto al cuerpo del programa de %%@
inicializaci\'on (l\'{\i}nea 6 del listado).
\end{itemize}

As\'{\i} mismo el cuerpo del programa de inicializaci\'on debe ir de preferencia en la %%@
direcci\'on 100h del segmento F000h, realizando esto como se muestra en la l\'{\i}nea 4 del %%@
siguiente listado parcial:

\begin{listing}{1}
CODIGO SEGMENT 'Code'
   ASSUME CS:CODIGO, DS:CODIGO, ES:NOTHING, SS:NOTHING

   ORG 0F0100h

PROG PROC NEAR  
; el cuerpo principal se coloca aqui...
\end{listing}

Esto es porque generalmente el ensamblador rechaza direcciones de inicio de un programa ubicadas %%@
en un desplazamiento menor que 100h, dado que los programas para producir c\'odigo binario lo %%@
generan para m\'aquinas IBM compatibles.

Un programa en ensamblador usado en SCIP tiene la siguente estructura:

\begin{verbatim}
CODIGO SEGMENT 'Code'
   ASSUME CS:CODIGO, DS:CODIGO, ES:NOTHING, SS:NOTHING

   ORG 0F0100h

PROG PROC NEAR  

; el cuerpo principal se coloca aqui...

   ORG 0FFFF0h

 START:                 
; Inicializar CS y saltar al inicio del programa
   MOV AX,0F000h
   MOV CS,AX
   JMP PROG
PROG ENDP
CODIGO ENDS
   END PROG
\end{verbatim}

Una vez que finalize el programa principal, se podr\'{\i}a poner un salto hacia el inicio del %%@
programa nuevamente, o un ciclo infinito que hiciera que la ejecuci\'on no saltar\'a a %%@
direcci\'ones err\'oneas. Esto depender\'{\i}a de cada aplicaci\'on.

Lo que conforme el programa puede ser cualquier instrucci\'on del 8088, siempre y cuando:

\begin{enumerate}
\item No se usen interrupciones de MS-DOS o del BIOS de una PC.
\item No se usen instrucciones de entrada o salida (a pantalla, por ejemplo).
\end{enumerate}

%----------------------------------------------------------------------------

\subsection{Programaci\'on de las memorias EPROM.}
\label{Subsection:progmem}

Las memorias EPROM utilizadas en el sistema son las Texas Instruments 27C512. Estas memorias %%@
pueden ser borradas con luz ultravioleta. El tiempo de borrado generalmente no excede los 5 %%@
minutos, aunque en caso de ser memorias viejas, pueden requerirse algunos minutos m\'as.

Estas memorias se pueden programar electr\'onicamente mediante un programador de memorias. Para %%@
las pruebas del sistema se utiliz\'o un programador AllMax, pero en cada caso se debe consultar %%@
la documentaci\'on del programador utilizado para determinar el procedimiento de %%@
programaci\'on. Generalmente el formato aceptado por el programador ser\'a Intel Hexadecimal o %%@
``Binary''.

Una vez programadas las memorias, \'estas deben ser colocadas en las bases correspondientes a su %%@
n\'umero: la EPROM con la rutina inicial en la primera base (ROM1, ver Figura %%@
\ref{Figura:partes}) y la segunda memoria en la base ROM2. La muesca en las bases de wire-wrap %%@
del prototipo indican la posici\'on de la terminal 1 de los circuitos integrados. Adem\'as como %%@
gu\'{\i}a para la colocaci\'on se puede ver la posici\'on de las terminales de tierra y %%@
voltaje en la parte inferior (alambrado) de la tarjeta prototipo.

%----------------------------------------------------------------------------
